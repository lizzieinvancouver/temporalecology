\documentclass[11pt,a4paper,oneside]{article}
\renewcommand{\baselinestretch}{1.4}
% \renewcommand*{\thefootnote}{\fnsymbol{footnote}}
\usepackage{sectsty,setspace} 
\usepackage[top=1.00in, bottom=1.0in, left=1in, right=1in]{geometry} 
\usepackage{graphicx}
\usepackage{epstopdf}
\usepackage{amsmath,latexsym,amssymb,wasysym}
\usepackage{natbib}
\usepackage{lineno}
\usepackage{todonotes}

\begin{document}
\bibliographystyle{/Users/Lizzie/Documents/EndnoteRelated/Bibtex/styles/amnat}

\noindent Title: \emph{Temporal ecology in a shifting environment}\\
\\
% Building the discipline of temporal ecology
% Temporal ecology and the future of integrative climate change research
% Temporal ecology in a changing world (from Kendra, but I do not like this, I'd prefer 'nonstationary world' if we're doing world or 'shifting' instead of changing
% Temporal ecology in the anthropocene
% Temporal ecology in shifting landscapes
\noindent E. M. Wolkovich,$^{1,2,3}$* B. I. Cook$^{4,5}$, K. K. McLauchlan$^{6,7}$ \& T. J. Davies$^{8}$\\
\\
\noindent \emph{$^{1}$Biodiversity Research Centre, University of British Columbia, Vancouver, BC, Canada; $^{2}$Arnold Arboretum, Boston, Massachusetts, United States of America; $^{3}$OEB, Cambridge, Massachusetts, United States of America; $^{4}$NASA Goddard Institute for Space Studies, New York, New York, United States of America; $^{5}$Ocean and Climate  Physics, Lamont-Doherty Earth Observatory, Palisades, New York, United States of America; $^{6}$ Department of Geography, Kansas State University, Manhattan, Kansas, United States of America; $^{7}$ University of Oxford, Merton College, Oxford, United Kingdom $^{8}$Department of Biology, McGill University, Montreal, Quebec, Canada}\\ %OX1 4JD for Oxford

\noindent *corresponding author: wolkovich@fas.harvard.edu, phone: 1.603.667.5099\\

\noindent Word count (total/allowed): 5876/7500 in main text plus 1234 in boxes currently, XXX/XXX in abstract % Ideas and Perspectives: maximum of 7500 words (main text), 10 figures, tables or boxes, and 80 references.

% \newpage
\begin{abstract}
\end{abstract}

\newpage
\linenumbers
\noindent {\large {\bf Introduction}}\\
Thirty years ago a transformation in ecological thinking was underway, precipitated by questions of how anthropogenic habitat loss and fragmentation impacted populations, communities, and ecosystems. Addressing these questions required ecologists to work at 
scales far larger than their traditional plot sizes, statistical methods and theories allowed; it required integrating perspectives and methods from other disciplines (e.g., geography and evolution) while building upon and developing a body of theories (e.g., island biogeography, metapopulation) and concepts (edge effects and corridors). The rise of spatial ecology has lead to important advances in both applied and basic research. Over time the field has tackled major ecological questions by building from single-species metapopulation to multi-species metacommunity models \citep{Pillai2011} and from local to biogeographical scales \citep{bell2001}. As the field has matured, a suite of dedicated journals  (e.g. \emph{Diversity and Distributions, Journal of Biogeography}) have provided a forum for the exchange of ideas and cross-pollination between otherwise disparate ideas and disciplines. %\footnote{If possible and quick would be nice to add something about how this came alongside lots more spatial data, and the computing to handle it, though not quite sure what drove what there. But, whatever, we're missin the data angle here. Though I bring it in below, which I think may be enough.}\\

In the midst of this transformation, anthropogenic forces have also fundamentally shifted the temporal dynamics of most systems. From arctic to temperate biomes climate change has produced extended growing seasons---fundamentally altering how organisms in these habitats experience time, and challenging ecologists to make predictions of how such temporal shifts directly and indirectly alter populations and communities. [Put phenology thing here?]\\

Much as questions related to habitat fragmentation pressed ecologists to work at larger spatial scales, climate change and related issues have challenged ecologists to better understand temporal dynamics, over longer time scales, facilitated by improved integration of perspectives from evolution and paleobiology. Together, these shifts have rapidly yielded data at larger scales than previously-available, with climate change in particular precipitating the identification and study of a wealth of long-term datasets in addition to those already available. For example, the study of phenology---a discipline with over a 1000-year record of study---has received With increasing availability of long-term data, creeping timescale issues have arisen: population dynamics that appear more complex when examined in longer time series \citep{Ziebarth2010}, selection that weakens when integrated over longer studies \citep{Hendry1999,schoener2011}. Together, we argue, these factors have resulted in a current need for researchers to reinvest in temporal ecology---to coalesce related concepts and theories as well as develop new theories and methods to guide further data collection, and to revisit fundamental questions of how temporal processes may structure ecological systems, from populations to ecosystems.\\

\noindent {\large {\bf Temporal patterns \& scales}}\\

Similar to space, time---in ecology and across disciplines---is populated by conspicuous patterns. We focus on four major ones here that are common in ecological systems: cycles, trends, noise, and events. Tidy, seasonal time series can be decomposed into cycles---recurring fluctuations (a fluctuation is rising and falling variation) that often have a consistent period---and trends, or the longterm movements of the time series (Figure 1). Trends represent one type of nonstationarity. Stationarity---the concept of any process with a probability distribution that does not change when shifted in time---is a major assumption of many time series methods and---we argue---ecological concepts and theory. Stationarity allows for extensive variation about a mean including possibly cycles and noise, but requires an unchanging underlying distribution such that there are no trends in the mean, cycles or noise. Nonstationarity also allows for the possibility of cycles and noise but, in contrast, it requires shifts---in the mean, cycles and/or noise---across time. Recent climate change, at least partially associated with greenhouse gases \footnote{CITE new IPCC, does anyone have the correct citation information for this?}, represents an excellent example of nonstationarity (Figure 1): for many systems, climate change has resulted in an obvious trend of increasing mean temperature, with variance about this mean shifted equally \footnote{Ben, recommend citation?}. In addition to cycles and trends, time series almost always contain `noise.' Often operationally defined as the variation remaining once cycles and trends have been removed, noise contains the irregular portion of time series and often include events. Events are the messiest of concepts in time (see Box 1: The trouble with events) encompassing slow processes that trigger rapid transitions \citep{Foley2003} and relatively rapid occurrences that produce long-term shifts \footnote{Ben, recommend citation?}, as well as extreme events, which are generally defined as falling beyond a statistically-defined threshold \citep{ipccextreme2012}.\\

An additional similarity between space and time is the importance of scale in defining and understanding patterns and associated processes. Just as patterns may change when examined at local versus regional scales \citep[e.g.,][]{Fridley:2007ct}, trends may appear as cycles, and parts of cycles as singular events or noise, depending on scale. Even nonstationary processes can be effectively stationary depending on scale---highlighting that nonstationarity does not necessarily imply nonequilibrium, as equilibrium dynamics would depend both on the temporal scale of the process and the temporal scale of nonstationarity studied.\\

While sharing some similarities with space, time is unique in several important aspects. First and despite great novels and films to the contrary, humans cannot directly manipulate time. While researchers have manipulated space at small \citep[e.g.,][]{huffaker} and large \citep[e.g.,][]{Terborgh:2001bw} scales with varying complexities, time can only be manipulated indirectly. Ecologists may adjust the timing of species' interactions \citep{Yang:2010cq} or underlying drivers of temporal processes, but they cannot fundamentally alter time to test its role in structuring systems. Next, time is directional. While space may have directional patterns (e.g., altitudinal and latitudinal trends) it is possible to view spatial patterns from almost endless directions and return to a place multiple times. In contrast, temporal patterns have a singular direction. Once an event has unfolded all following patterns and processes may be impacted by it without any temporal recourse to return to it or examine it in another direction. Finally, humans experience only a very tiny snapshot of time. Thus, while ecologists may cover the entire globe to map spatial patterns, temporal patterns governed by very short or very long timescales are inherently difficult for humans to study. For mortal ecologists temporal dynamics playing out over timescales longer than a career have never been easy. In the words of \citet{wiens1986},``[w]e get only a brief and often dim glimpse of the relevant processes.'' Together these unique aspects have shaped how ecologists have tended to view and approach time. \\

\noindent {\large {\bf Temporal patterns in ecology}}\\

Ecology has a long history of incorporating temporal dimensions into its research \citep{clementsbook}, with a special focus on fluctuations, cycles, and noise (Figure 1). Understanding how organisms respond to daily fluctuations in the environment makes up a large portion of study in behaviour \citep{macarthur1958} and physiology \citep{Lambers:2008jb}. At longer timescales population ecology has focused on between year dynamics, with foundational work aimed at understanding the underlying factors that drive many population cycles---from cicadas to lynxes \citep{Krebs2001,Yang:2004zd}. Extensive bodies of work have been built on studying the attributes, behaviours and possible drivers---both internal \citep{Gurney1985} and external \citep{Hallett:2004oj}---of cycles in densities with extensions to how ergodic properties of ecological systems may produce complex dynamics, such as chaos \citep{May1976}. Fluctuations have been heavily studied with regards to species coexistence, where fluctuations in the environment have variously served to speed collapse towards one species or promote coexistence, depending on the theoretical framework \citep{Chesson:1997dz}. Relatedly, the random variations outside of considered fluctuations---temporal `noise'---have also received extensive study by both population \citep{Ripa1996,Kaitala1997,Bjornstad:1999kl} and community ecologists \citep{Chesson:2000vd}. Indeed, classical community ecology has used both intra- \citep{parrish1979,Albrecht:2001id} and interannual \citep{Chesson:1997dz} timescales to explain coexistence via temporal partitioning or small-scale differences in species' responses to a temporally-variable environment \citep{macarthur1958,Hutchinson:1961ui}. \\

Over recent decades ecology has shifted increasing focus to nonstationary patterns and events, which are often coupled. Concepts of nonstationarity have been long-studied in certain arenas of ecological thinking: succession---changes in the structure and function of ecosystems over time---is fundamentally a prediction of nonstationarity and one of the basic concepts of ecology \citep{clementsbook,gleason1926}. This nonstationarity is in turn invoked across studies of community \citep{Levin:1992rg} and paleoecology \footnote{Kendra, please suggest ref} that explain diverse landscape patterns via events---often disturbances---that reset successional clocks. How systems respond to events is also foundational across ecology in studies of resilience, and other definitions of stability \citep{Grimm:1997}. Concepts of systems with multiple stable states \citep{yang2010}, another version of nonstationarity, also generally cite events that may shift systems from one state to another. Such events may be either rapid \footnote{Ben, suggest ref?} or slow forcing events \citep{Foley2003}, such as recent anthropogenic climate change. Additionally, many other anthropogenic forces affecting ecological systems---for example, fire regimes that shift due to human ignition sources, damn-building phases in step with regional and country-level development---are nonstationary. In contrast, many underlying assumptions of ecological theory and basic approaches are not.\\

\noindent {\large {\bf The modern discipline of temporal ecology}}\\

Complexities in ecological time series (Figure 1, Box 1) and feedbacks inherent in biological systems mean that---while ecologists have long studied temporal dynamics---they have not always fully embraced the diversity of temporal issues rampant in ecology [possible cite JD's tipping point papers]. This is understandable given the unique attributes of time, methodological constraints and, until recently, dearth of available time series data. However, climate change, and other aspects of how humans have shaped ecological systems, have highlighted how omnipotent temporal dynamics are. We argue that ecological research would accelerate with a renewed, more constant temporal focus---across and between subfields and related disciplines. Such a focus should be applied in both approaches to methods and in building concepts and theory.\\ % add nonstationarity above?

\noindent \emph{Building depth in approaches \& methods}\\
Across many areas of ecology---with some notable exceptions---there is often a hidden rift between the relevant timescale for a given ecological question and the timescale of methods applied to address it. Consider, for example, two common types of experiments: predator exclusion studies to estimate the impact of predators on prey abundances and manipulations of biodiversity to estimate ecosystem functioning consequences. In the first example the appropriate timescale of such a study would include an estimate of the timescales at which the predator affects its prey: this would in turn depend on the timescale of the predator and its population cycles, as well as the timescales of all species of its prey and their---most probably diverse---generation times and population cycles. Layered on this would be consideration of the timescale that predators interact with their prey; for example, predators may feed on certain prey only when other prey are rare and such periods may occur infrequently---requiring longer studies to adequately estimate the full response. In the second example, consider a manipulation of richness in a plant community and its effects on nutrient cycling. In this case the appropriate timescale would depend strongly on the processes behind nutrient cycling including decomposition of above and belowground plant tissue, including relcacitrant fractions of tissues, which---even for herbaceous plants---can last 10 years or longer \footnote{Kendra is this true? And if so, send ref}, plus turnover of labile and recalcitrant soil fractions \footnote{Kendra please fix}. \\

Very quickly it becomes obvious that adequate answers to many basic ecological questions would require studies on the scale of a single ecologist's career or longer. Given this, alongside the timescale of grant cycles it is not surprising then that many experiments and other general approaches in ecology make due with relatively short-term studies often an order, or several orders, of magnitude smaller than would be ideal. One option to ameliorate this issue is to suspend the tacit interest in equilibrium or quasi-equilibrium dynamics in the examples considered above and focus instead on transient dynamics. However, this would mean abandoning a wealth of ecological theory based on some version of an equilibrium state. Further, many applied questions require a longer time horizon than short-term dynamics provide: for example, restoration of a degraded habitat is not generally considered a success if it follows an expected trajectory then reverts to its degraded state after several years. An alternative solution is to work in systems where the relevant timescales fit neatly within grant cycles. This is certainly a valuable idea, but unnecessarily precludes most systems from studies on fundamental ecological questions. Instead of such hobbling, ecologists could more readily employ methods that attempt to ameliorate timescale issues or, at a minimum, better highlight the scales at which current methods provide insight.\\

Several major approaches in ecology address temporal scaling issues. The first and perhaps most-widely applied method substitutes space for time (chronosequence, CITE Jenny 1941). Research in long-term ecosystem development relies on this approach [Jenny 1941], examining how the state factors of climate, organisms, relief, and parent material vary across space to understand long-term shifts in ecosystem properties, such as net primary productivity \citep{Wardle:2004wb}. This type of approach makes several major assumptions---including stationarity in the trajectory of ecosystems across space (see `Contingency' section below)---and requires extensive effort to locate relevant sites and pinpoint their temporal dimension, but provides tremendous power to ask questions on timescales far longer than the full lifespan of the field of ecology and, perhaps most powerfully, to project forward. Another very basic approach is the comparison of methods, sometimes abstracted into experiments, observations, long-term observations and modeling \citep{Carpenter:1992hk}. Experiments are often conducted on the shortest timescales---from days to weeks in the lab, to weeks and years in the field---and, as discussed above, may only capture transient dynamics. They allow, however, generally the most powerful tests of mechanisms and, relatedly, major insights \footnote{Anyone have a suggested reference here? I'd like one that makes experiments look good}. Such powerful tests are buoyed by comparisons with observational data, even short-term. For example, a experimental study of the impact of an invasive species by manipulating its density may find certain effects: showing similar effects along a gradient of low to high abundance of the invasive species provide support that results are not due to experimental artifacts. Showing congruence with long-term data is a further boost and using modeling to extend the scale of the findings---possibly incorporating important longer-term dynamics such as climate cycles---can greatly improve the utility of the original experiment as well as most probably providing additional insights and new predictions to test. Related to this, a greater incorporation of a longer-term temporal dimensions to short-term studies of long-term processes may come from incorporating perspectives of disciplines traditionally focused on long timescales. For example, paleo ecology has provided major insight and complements to modern studies of current climate change \footnote{Kendra, please fix}. While something about eco-evo could go here \footnote{JD, please fix}. \\

One additional simple method to clarify the temporal dimensions needed to address particular ecological questions is to fully recognize the timescales at play. This means giving all the relevant known timescales---these include the generation times of the study organisms, frequency of disturbance, as well as the period of climate oscillations---then placing the study in the relevant part of these cycles. For example, a study of spiders on shrubs in a semi-arid system where precipitation covaries with the El Ni\~no Southern Oscillation (ENSO) should mention when during the general reproductive and population cycle spiders were sampled, how long since the most recent fire disturbance, what phase of ENSO the study year(s) included and what the precipitation amount was in regards to long-term averages. Responses may be quite different in high versus low-rainfall years and trajectories of experiments may vary depending on the rainfall of the year the study started. This extends to studies of multiple systems, where timescales may vary dramatically. Thoughtful discussion of temporal issues---or at a minimum giving relevant temporal dimensions---can enhance a project with cascading benefits for the field of study. Indeed, major debates in ecology have often arisen from hidden timescale issues. For example, in studies of trophic cascades, freshwater systems with rapid turnover times often yielded results in accordance with equilibrium expectations, while studies in terrestrial systems---where experiments were much shorter than many of the manipulated species' generation times---often yielded contrasting results due to transient dynamics \citep{Cebrian:2009hg}.\\ 

The above-described approaches are not new in ecology. It is not that ecologists do not use these methods, it is that they have been unevenly applied across time and across subfields such that the inherent complexities time introduces to ecology have not always been transparent. A more even application of these approaches would benefit ecology, but will require better training of students in the above approaches and higher expectations of approaching all ecological questions with a more overtly temporal perspective. This will mean continually cross-checking approaches of varying timescales, modeling studies to extend beyond currently available data, and more integration of disciplines that have sometimes worked separately because of their underlying disparate timescales. Such approaches will inherently result in improved studies and the collection of more long-term data.  With this underway, there will be in turn a greater need for ecologists trained in time-series analysis and more complex statistical and conceptual approaches that can handle nonstationary data (see Box: A taster of time series methods). Such a shift is especially needed given the current rates of nonstationary processes, such as climate change, operating in many systems. \\

\noindent \emph{Building depth in concepts \& theory}\\
Building depth in how ecologists approach temporal dimensions of ecological questions and their implementation is a good start, but ecologists need to revisit concepts and theory from a stronger temporal focus---one that includes cycles, events and nontationarity and engages perspectives from across ecology and other relevant disciplines. This means both more clearly recognizing temporal aspects of theories and concepts and diversifying the temporal dimensions and patterns such theories consider. We review below how renewed efforts to answer fundamental questions about about cycles and temporal scaling issues could greatly advance basic and applied ecology, while, at the same time a much stronger focus on incorporating events and nonstationarity into theory could form the basis for a new mode of ecological thinking: one that does good stuff. [FIX] \\

\noindent \emph{Subsection: Mixing pattern \& process: Cycles}\\ % or,  Recurvsive study of cycles
Temporal dynamics represent a rare example in ecology of where pattern may drive process. Ecology is generally a field focused on understanding the processes that produce the diversity of observed patterns. Such patterns are assumed to be the product of one (or more) mechanistic process and, thus, ecologists often build theory that assumes a static temporal background to yield predictions of observed patterns. In contrast, time can---independent of ecological processes---exhibit its own patterns, often related to geological or climatic processes. This duality of causation: where ecological theory can produce temporal patterns from static backgrounds, but also where such backgrounds may have their own complicated patterns even before the introduction of ecological systems is a well-known issue throughout much of ecology. Most prominently, understanding to what degree regular fluctuations or cycles in ecological systems are shaped by external temporal patterns or are driven by ergodic properties of populations and species interactions has been an area of active research in ecology since the beginning of the discipline. \\

For decades ecologists have struggled over how to differentiate external forcing from internal autonomous cycling, and, relatedly, how much do fluctuations generate all other fluctuations. While these are foundational questions in ecology (Allee 1949), they remain effectively unanswered. Copious modeling studies have successfully produced myriad complex patterns from models using only ecological dynamics, and no external forcing \citep{May1976}. However, the implications of these findings to field ecological systems are themselves nuanced. For example, the explicit time dependence of populations assumed in many models encompasses an implicit time dependence, which could occur via  multitude of factors including population dependence on temperature \citep{Gurney1985}. \\ %Applications of such models outside of many lab and crop studies has shown time and time again that ecology is complicated \citep{Bjornstad2001}. 

Studies approaching the questions from more purely empirical angles have not, however, reached any concordance either. Building from empirical field data, studies testing for links between population densities and climate have found varied success. While recent work building on long datasets and improvements in our understanding of how large-scale and long-term climate oscillations control weather have found strong correlations between population outbreaks \citep{Stige2007} and declines \citep{Coulson2001}, others have not \citep{Knape2011}. \\ %Certainly, studies finding links have not always clearly considered alternative drivers.

Despite this discordance within both the camps of modeling and empirical approaches, the current availability of time series data and modeling approaches we believe offers a major opportunity to make progress on when and to what degree ecological cycles are externally or internally driven. To date studies that blend models and time series field data---let alone those that test alternative models with direct contrasting predictions---are still few \citep{Bjornstad2001,Knape2011}.  It seems, however, that current ecological theory is complicated enough to produce the full myriad of observed complex patterns in nature. At the same time, advances in understanding weather patterns \citep{Bjornstad2001,Weart2013} have provided a much greater diversity of ways to link populations dynamics to climate. Thus, the challenge must turn to developing methods and predictions of theory that provide actual tests of external versus internal forcing, or---as will most probably be the case---a mix of the two \citep{Benton2006}. Such work would advance one of ecology's most fundamental, and unanswered, questions, while providing strong inference in how forecasted climate shifts will impact population dynamics.\\

\noindent \emph{Subsection: Hidden temporal dimensions in space}\footnote{Jonathan, please fix/add/rescue, otherwise may remove section entirely as Kendra is not (yet) a fan of it. No pressure though.}\\
Recent advances in temporal ecology have come from the rise of spatial ecology and its emphasis in recognizing and understanding hidden dimensions in ecological models and theory. Decades of efforts in understanding the consequences of when space is made explicit have resulted in a return to the importance of temporal dimensions of many spatial models. For example, island biogeography theory predicts species richness based on several basic spatial metrics---but temporal dimensions of the controlling processes---immigration, extinction and speciation---also clearly govern predictions of species richness (CITE WIENS). Similarly, disease models have advanced through incorporating both spatial and temporal models of traveling waves \citep{Grenfell:2001ox} as disease prevalence varies both with population density and temporal fluctuations in that density \citep{Grenfell:2001ox} and climate \footnote{JD, please suggest ref}. \\

Such advances represent, however, only a small foray into the applications of fully embracing the interconnectedness of spatial and temporal dynamics in ecology. Coexistence theory---long stymied by models that required \(n\) different axes to produce \(n\) coexisting species alongside empirical examples of many co-occurring species that appeared quite similar when examined from one snapshot in the lab or field---advanced several football fields beyond their general stronghold at the 10-yard line when the role of variability in species responses to the temporal dimension was re-examined \citep{Chesson:1997dz}. Under the storage effect model \citep{Chesson:1997dz} species with identical nutrient uptake curves, identical predators and predation rates can co-exist via small differences in how they respond to temporal variability in their environment. Given the tremendous diversity of temporal variation in most environments it should not be surprising that it would be a major axis on which to mute competitive forces and thus promote coexistence. Since its introduction the storage effect model has been ported to spatial dimensions as well---where species coexist via reduced competition from spatial variability. Tests for such models have found support separately for temporal \citep{Angert:2009} and spatial \footnote{Lizzie adds ref} storage effects in studied communities, but we expect most communities function based on a constantly shifting mix of the two mechanisms (in addition to fluctuation-independent mechanisms). For example, studies of community change during the Dust Bowl show how dramatically dominant species may decline to apparent local extinction while rare species rise to abundance \citep{Weaver1936}. Recent work, however, highlights the persistence of rare species throughout long time periods through landscape dynamics where microclimates maintain great species diversity \cite{Craine2012}. In such cases temporal storage effects are built on buffered population growth maintained by spatial dynamics; this is in addition to the more traditionally-considered temporal buffering via seedbanks or other long-lived lifestages of species. \\

Understanding how both temporal and spatial dynamics may control communities is an important step to climate change projections, but will critically require an additional layer of understanding how nonstationarity will shift these dynamics. The Dust Bowl was a major event for tallgrass plant communities---and typifies an important difference between many previously-studied climate events and the non-stationarity of current anthropogenic climate change. Following the Dust Bowl as climate returned to a state similar to that before the Dust Bowl, many plant communities also eventually returned to species assemblages similar to those before the Dust Bowl \citep{Weaver1936}. This is a rough prediction of the storage effect model, which assumes underlying stationarity in the environment and species' responses to it. Climate change, however, is specifically a nonstationary event and predictions of what happens to communities under nonstationary environmental dynamics are not well-developed (see next section on `Contingency'). The consequences of nonstationarity in temporal dimensions in most ecological models are unknown but critical for projection. For example, island biogeography makes predictions assuming stationarity in the mainland population, yet most mainland populations in recent times are rarely stationary. Additionaly, robust projections of climate change impacts on populations and species will require a nonstationary adjustment to the most classic spatial metaphor for a temporal process: adaptive landscapes. Ignoring species interactions and considering only shifts to the abiotic environment, climate change has resulted in rapid and effectively continuous shifts to most populations adaptive peaks and valleys. Climate change has thus highlighted how rapid evolution may be and bringing it firmly into an ecological timescale, but theory to how such nonstationarity may affect evolutionary outcomes has not kept pace \footnote{Lizzie add ref}.\\

\noindent \emph{Subsection: Contingency: How persistence, irreversability and nonlinearities define communities and ecosystems}\\

While nonstationarity as a term may not be a catchphrase in ecology today, it pervades certain arenas of the field and underlies some of the discipline's oldest concepts, and debates. One of ecology's most basic concepts, succession---the changes in the structure and function of communities and ecosystems over time---is a concept of nonstationarity. The classical debate of how predictable succession may be---pitting Clementsian versus Gleasonian versions of nature against one another---provides an important additional layer. In the Clementsian version of nature, succession is predictable. The classic story from primary succession is exemplified in work from newly-deglaciated surfaces in Glacier Bay, Alaska. It involved a temporally-predictable sequence of four vegetation types: pioneer, Dryas, alder, and old-growth spruce and hemlock forests [W. S. Cooper]. Over this sequence, ecosystem properties changed such as an increase in organic N content and a decrease in soil pH \citep{Chapin1994}. Biotic interactions among species such as facilitation and inhibition were considered to control the rate but not the nature of succession on these sites. Thus, a gradual change from abiotic to biotic control of ecosystem properties was occurring over successional time, with the inertia of soil organic matter providing the key factor for this transition \citep{Milner2007}. In the Gleasonian version, however, successional pathways are rarely as predictable. Returning to Glacier Bay, dendrochronological reconstructions of the vegetation history at different sites of the Glacier Bay sequence revealed at least three different primary successional sequences \citep{Fastie1995}. The oldest sites had early establishment of spruce and hemlock (the early stages were missing or short). The middle-aged sites had a prolonged alder phase, and the youngest sites had a cottonwood phase in addition to the four classic phases. The revised model of primary succession needed to include the roles of parent material, proximity to seed source, and lack of constant climate history across sites. Secondary succession, which is commonly observed after disturbance events, has been even more difficult to predict temporally. \\

This shift in perspectives on succession is one example of ecology's maturing view of the complexity of community assembly (Figure 2 `previous'). Early views of succession effectively assumed a simple environmental filter where climate was implicitly assumed to be stationary background noise, never providing any means to shift or alter trajectories; layered on this was the Hutchinsonian niche model where species coexisted via separate hyperdimensional moulds of resource use (which were often implicitly assumed to be effectively constant in space and time). A more modern view of succession includes an environmental filter that considers climate means as well as regular oscillations and events that may reset succession (Figure 2 `recent'). While researchers still sometimes use climatic niche distribution models to estimate environmental filters these is growing return to basic physiological limits that truly define the environmental filter for most species. Such physiological limits are most often defined by climate events and it is here that climate projections are moving. \footnote{Lizze: Add in Ayres beetle example.} Layered on this is increased recognition of the important variability in climate to coexistence: the Hutchinsonian niche is now one considered mode of coexistence, with increasing research in Chessonian coexistence mechanisms such as the storage effect highlighting the importance of temporal variability in shaping the biotic filter of community assembly. Chessonian coexistence has, in some ways, blurred the lines between the environmental and biotic filters of community assembly. \\

The current evolution of community assembly theory is incorporating nonstationarity in the environmental and biotic filters (Figure 2 `current'). Already studies in paleo and disturbance ecology highlight that multiple successional pathways, with distinctly different end trajectories are possible. For example, two contrasting successional pathways were identified in a study of regeneration dynamics of ponderosa pine (\emph{Pinus ponderosa}) forests under extreme climate conditions and a human-altered fire regime in the American Southwest \citep{Savage2013}. After high-severity, stand-replacing fires in the late 1940s to the mid-1970s, former ponderosa pine sites became either: (1) hyperdense ponderosa pine stands vulnerable to further crown fire, or (2) nonforested grass or shrub communities \citep{Savage2005}. Drought conditions in the years immediately following a severe fire inhibited ponderosa pine regeneration, which effectively closed the climate window favorable for regeneration of that plant community. It was the synchronous occurrence of high-severity fire and drought that was the key predictor in this system \citep{Savage2013}. We argue a major furture challenge will be to understand how much nonstationarity---due to a variety of processes including human modification of climate and disturbance regimes---may often contribute to multiple successional trajectories, and how the interplay of events and nonstationary may alter trajectories (Figure 2 `current'). Layered on this is the challenge of adapting coexistence theory to nonstationary environments. Climate change and the rising study of Chessonian coexistence mechanisms, where communities are built on small differences between species in their responses to a variable but stationary environment, leads to the critical question of what happens to communities and populations built on such coexistence mechanisms when the environment switches from stationary to nonstationary. \\ 

Modern ecology, with some exceptions, is effectively Gleasonian.  It is infused with a continually broadening recognition of system states that are defined by nonlinear trajectories and irreversability \footnote{Add good ref}, of the persistence by which factors affect communities and ecosystems, and an overarching recognition that historical contingency is more common than predictable sequences over time. Studies of community and ecosystem stability (see Figure 2), paleoecological systems and modern disturbance ecology have provided foundational work on the role of contingency in driving ecological systems. The challenge to ecology is now to build theory that incorporates contingency in such a way that studies can provide more robust tests of how contingencies operate: for example, the role of multiple or compound disturbances in altering trajectories and whether environmental nonstationarity may make regime shifts more common by effectively removing the underlying environmental track a system was previously on (Figure 2 `current'). The importance of historical contingency in ecology should not in any way suggest that systems are less predictable, it only highlights what ecologists already know so well---that ecological systems are complicated and accurate prediction requires careful knowledge of many driving factors. \\

%Some questions: What has to happen for a process to be irreversible? Related, Does nonstationarity make true regime shifts more common (harder to get back on track when track has shifted)?
\noindent \emph{Subsection: Temporal scaling of ecological processes}\\
Temporal scaling issues have been extensively discussed in many areas of ecology, including the struggles of incorporating fast and slow processes in models of ecosystem dynamics \footnote{Lizze adds ref} and temporal community coexistence via with year dynamics (e.g., phenology) versus between year dynamics \citep{Chesson:1997dz}. Many of ecology's most vociferous debates have routinely hinged on temporal scaling \citep{wiens1986} and many foundational questions in the field are still centered on the difficulties of dealing with time (e.g., How do slow and fast processes combine to produce observed temporal dynamics? Do biotic versus abiotic processes govern different timescales? How does generation time matter and what not?). Our message here is therefore, insanely un-new (and thus will be brief). \\

Climate change has, however, refocused ecological thinking on temporal scaling. It has provided a major new impetus to revisit fundamental questions whose answers have applications far beyond climate change: for example, understanding how environmental nonstationarity affects coexistence via the storage effect has implications for questions of how and when the storage effect may evolve and how species survive glaciation and other major climatic events. For, while anthropogenic climate change is a dramatic alteration to the earth's climate system, equally rapid shifts in climate on similar timescales have occurred before and equal shifts occur commonly in the environments of most organisms on far shorter timescales (Figure 3).\\

Today, ecology has an additional step-up in reapproaching questions of temporal scaling from greater integration with other disciplines. Just as temporal scales inherently define and link subfields in ecology (Figure 3) they also bridge ecology to climate science and evolution. This current advantage is also a challenge, however: common temporal concepts often developed within subfields and fields independently producing a more complicated lexicon than may be ideal (Table 1). We expect progress in temporal ecology will come from more integration that allows recognizing common terms and concepts. Such basic advances in temporal ecology will require better grounding in climate science, greater influence from studies of physiology, behaviour, biogeography, paleobiology and micro as well as macroevolution. \\ %Mechanistically understanding how diverse species respond to climate change will require basic advances in temporal ecology, including better grounding in climate science (Box 1), greater influence from studies of physiology, behaviour, biogeography, paleobiology and micro as well as macroevolution.\\

\noindent {\bf Conclusions}\\

Just as spatial ecology provided a forum to bring together similar lines of research from disparate fields and disciplines, a reinvigoration of temporal ecology catalyzed by climate change should provide an arena to build a cross-disciplinary field of temporal ecology. Whilst there has been a rapidly growing body of work on understanding how species and communities will respond to increasing rates of anthropogenic climate change, general theories and paradigms to shape and guide studies have yet to fully emerge.  We believe a new framework is needed that parallels efforts in spatial ecology---improving and developing methods, building on basic theories and concepts---but structured by time (evolutionary process, climatology, time series methods). This temporal focus will require integration across timescales, disciplines and methods. Recent advances across subfields in ecology---in incorporating environmental variability into coexistence models, bridging ecological and evolutionary timescales, revisiting the role of climatic events in setting range limits and in modernizing paleoecology \footnote{Kendra fix this if you think we should keep it}---all suggest modern ecology is up to the challenge. So raise a glass, three cheers, merry, merry all around. It's all goodwill here from Lizzie to the ecological nincompoops that I work with (oh shoot, slipped up there). \\

\noindent {\bf Acknowledgments}\\
We thank both Monsters and Men. Conversations with D. Schluter and M. Donohue made my life better and helped this manuscript. NSERC BRITE funding.

\newpage
\begin{small}
\bibliography{/Users/Lizzie/Documents/EndnoteRelated/Bibtex/LizzieMainMinimal}
\end{small}

\newpage
\begin{center}
\begin{table}
\caption{Temporal concepts (across subfields and disciplines, perhaps. Need to discuss if we will keep this. It could possibly work better if we coordinate it some with `taster of time series' methods.)} 
    \begin{tabular}{ | p{4 cm} | p{3 cm}  | p{3 cm}  | p{4cm} |}
    \hline
{\bf Concept & Discipline/field & Methods & References} \\ \hline \hline
lag effects/carry-over effects	& ecosystem/ community/ micro-evolution & autoregressive models & Osvaldo Sala's work; Betini et al. 2013 \\ \hline
succession & community & chronosequences & Gleason, Cooper, Fastie \\ \hline
biogeochemical trajectory & ecosystem  & chronosequence  & Peltzer 2010, Wardle 2004 \\ \hline
colour of noise  & population  & basic time series methods & \\ \hline	
regime shifts/alternative stable states/ multiple stable equilibria  & community/ population ecology/ [evolution here JD?]   & short time series (Lake Mendota), starting to get into paleorecords  & Martin Scheffer and crew \\ \hline
regime shifts  & ecosystem/ community   & breakpoints/ changepoints  & Monica Turner (?) 2010 McArthur paper in Ecology \\ \hline
extinction debts & population/ evolution & & 	JD: add references \\ \hline
synchrony/Moran effect & population/ community & & \\ \hline	
pulse vs. press & ecosystem & & \\ \hline		
bet-hedging & population/ community & & Venable \\ \hline
temporal niche & community	& & Gotelli/Chesson/ Hutchinson \\ \hline
    \hline
    \end{tabular}
\end{table}
\end{center}


\newpage
\noindent {\bf Box 1: The trouble with events} \\
\noindent \emph{To go with the text there will be 2-3 figures. The first is simple, and I was hoping Ben could send some data for this: a simple time series showing a frost event---I think this could just be a couple weeks of climate data from some system. Then the measles data with the breakpoint analysis (shown) and then we could add possibly one more where the issues in having a short time series and trying to identify drivers are clear (one option shown). }\\
\begin{figure}[h!]
\centering
\noindent \includegraphics[width=0.6\textwidth]{/Users/Lizzie/Documents/git/manuscripts/temporalecology/figures/events/eventsbox.png}
\end{figure}
\newpage
% Events box: move squishiness of term to main text. Transient vs continuous forcing and responses instead of ‘events.’ Think of dumping P in a lake — it’s transient forcing — the response could be either transient or continuous. Hurricane is transient forcing, measles is continuous!
Events are critical to the dynamics of ecological systems. Much research in ecology has focused on understanding how various events, especially certain disturbances (e.g., fires, ice storms, hurricanes), may fundamentally shift the dynamics of ecosystems (Mumby et al. 2007 example). In this conceptual arena events are external to a system (e.g., the calcium spike applied to watershed 1 at Hubbard Brook, see Figure 1) and studied as possibly important triggers. Events, however, may also be internal (e.g., the response of magnesium to the calcium spike, Figure 1) to a system.\\

The term `event' is a classic polyseme---taking on various somewhat ambiguously related different meanings depending on the user and context. In probability theory an event is a draw from a probability distribution, thus it can take on any value from a defined probability space. Such a definition may be accurate but is relatively useless for building a field focused on better understanding events. In most of its usages, however, the term `event' takes on more specific meanings. In ecology [and any other disciplines?], an event is often a property of a time series that cannot be easily decomposed into a cycle or clear trend, and may often show up statistically within the `noise' portion of many classical decomposition methods (see Figure 1). Beyond this events may take several forms, which we broadly categorize into three general types:
\begin{itemize}
\item \emph{Point events} are are most simply defined as a single fluctuation within a time series, outside of any cycles. They must thus be particularly high or low values compared to the rest of the time series (see frost example that we'll add). Point events, thus, depend greatly on the length of the time-series: what would be characterized as an event in a short time series may appear clearly as part of a cycle given a longer time series (e.g., one high rainfall year in a 3-4 year record in the US southwest may be be an event for that length of time series but appear as clearly part of the ENSO cycle given an multi-decadal time series), just as what may appear to be a trend in a short time series may be an event given a longer time series (could add an example if anyone has a good one). Additionally, the timescale of the time series affects detection of point events: sampling events that are too infrequent or average over long periods may mask variation and thus events in time series data. For example, models using average daily data or monthly climate data may miss important temperature swings beyond a study organism's tolerance level, which may be the critical link to understand the effect of temperature. Extreme events are a good example of point events: they represent high or low values in a time series as defined by some statistical threshold (often X , cite IPCC), but their definition shifts greatly depending on the length of the time series considered and the timescale of data collection (any excellent examples?). 
\item \emph{Shift events} are the time at which a dramatic shift in a system takes place: often a shift to a different mean value or changes in the variation of cycles. For example, the 1968 introduction of the measles vaccine in the United Kingdom produced a dramatic reduction in the peak values of the multi-year disease cycle: this external event thus produced a shift event in the time series. Determining the exact timing of shift events is often difficult depending on the system and nature of the event. The measles data highlight this: though the vaccine was introduced in 1968 (blue line) because it shifted the peak values of cycles, identifying its effects on the system would not be visible until the next high peak in the cycle is due. Therefore it is not surprising that a simple Chow breakpoint analysis identified 1971 (red line) as the year the time series properties shifted: this may be because the nature of the process did shift fundamentally in 1968 but this shift was not visible until 1971, or because there were lag effects that delayed a major shift in the underlying process until 1971. In this case the timing of the external event relative to where in the cycling the time series is may also impact inference: if the vaccine was introduced just before a peak period then the window of uncertainty as to whether the system had shifted would be narrower. The inherent nature of certain time series and the shifts that may occur within them---especially shifts in cycles---means it may be impossible to tie events together at the same moment in time. [As with the whole document, but especially here: other examples and comments welcome.]
\item \emph{Slow forcing events} are stuff like the Green Sahara example. [Ben, can you write more here? We can note that because of temporal scaling the term slow is clearly relative to the question of interest and to the time series of analysis. Or we can debate a better name for these sorts of things.]
\end{itemize}
\\
These three broad type of events all highlight how context-dependent the detection and definition of events are. The length, timescale and beginning and end points of a time series affect whether and how events are noted. Layered on top of this time lags in responses to events can also make detection difficult [Ben or someone: example or anything to add here?]. Thus, in ecology understanding events raises multiple challenges and, further, requires caution and acceptance of often inherent limitations. Though ever increasing, time series data of any great length in most ecological field systems are rare and thus researchers must constantly question how the attributes of their time series data (length, timescale and beginning and end points) affects inference. For example, it may be impossible to strongly link certain events to regime shifts using only time series data when the data begin with a system that is clearly already varying (Mumby et al. 2007 example). \\
 
Additionally, certain events may be best classified as Black Swan events. A Black Swan event is defined by two components: (1) that it has dramatic effects on the system, but is extremely rare, such that (2) it is effectively impossible to predict using current methods. These two components lead to the third aspect of Black Swan theory: owing to a combination of their unpredictability and their large impact on the system there is a strong tendency to believe such events can be predicted---when, instead, by their extreme rarity and unpredictability this is generally impossible. Such limitations may be common in many systems and statistically rare events in time series data are especially difficult: as discussed above, the definition of `rare' or `extreme' must shift with the data---extreme values almost always become less extreme with longer time series. We mention Black Swan theory therefore not to emphasize the unpredictability of ecological systems, but instead to caution that researchers recognize inherent limitations in time series data and methods, and in our perceptions of how well---given such limitations---we may predict rare events. \\

\noindent {\bf Box 2: A taster of time-series methods}
6-12 two-panel panels (underlying time series data, then output of method I think). Includes discussion of frequentist, Bayesian, Monte-Carlo. \footnote{I am feeling overwhelmed by this figure a little and want to chat with Ben about it some time soon. Then I/we will get on producing it here. }

\clearpage
\newpage
\begin{figure}[h!]
\centering
\noindent \includegraphics[width=1\textwidth]{/Users/Lizzie/Documents/git/manuscripts/temporalecology/figures/ts_babies/tspatterns.png}
\caption{The tidiest, most classical time series data (raw data shown on top: \(CO_{2}\) from Mauna Loa (left) and magnesium from Hubbard Brook Watershed 1) can be decomposed by age-old traditional methods into cycles (often seasonal) and trends; what remains after this is 'noise'---aka, the `error' or `random' component. Research in population and community ecology has traditionally focused on cycles and noise, but understanding how trends impact ecological systems is currently a good deal important. Cycles and noise are, by far, the best studied temporal patterns in ecology---from daily and seasonal cyclical patterns of photosynthesis and respiration of plants \citep{Lambers:2008jb} and ecosystems (CITE-ask Kendra) to cyclical population cycles of predators and prey \citep{Krebs2001}. [In this caption we may also discuss how these three major attributes of time affect ecological processes: cycles (and species interactions with cycles), events and trends---both of which may trigger regime shifts. Need to add labels to Y axes, and point out some key features in each dataset, plus add references.] \label{decompfig}}
\end{figure} % Classical decomposition of time-series data breaks out cyclic (often seasonal) components, trends and leaves behind noise. 

\begin{figure}[h!]
\centering
\noindent \includegraphics[width=1\textwidth]{/Users/Lizzie/Documents/git/manuscripts/temporalecology/figures/assembly/assembly.png}
\caption{Conceptual figure: Goal of this figure if we can make it work is to review major theory based on time in ecology---succession---and its underlying stationarity assumptions (top), but also point out other possibilities. In the middle, climate with some cycling to it and how this may shift ecosystem development, and bottom, how non-stationarity may induce regime shifts. [I clearly need to work on this caption more.] \label{successfig}}
\end{figure}

\newpage
\begin{figure}[h!]
\centering
\noindent \includegraphics[width=1\textwidth]{/Users/Lizzie/Documents/git/manuscripts/temporalecology/figures/ecoevoclimate/climateEcoEvo_alltogethernowPlus.png}
\caption{Timescales of climate events and processes versus ecological processes and traditional methods. The concept of time is foundational to all ecological study. While the emphasis on time has varied with the development of ecology and between various subdisciplines of study---at varying periods and in vary fields, being treated implicitly, explicitly or a central guiding framework---it has always served as the scalar link between diverse fields of study. At the shortest timescales physiological and behavioural studies have focused on processes that operate on minutes to days or months and often repeat themselves over longer timescales (e.g., shifts in photosynthesis or metabolic rates that vary predictably each day with lightcycles). Finally, time forms the backbone of eco-evolutionary theory---where evolution, assumed to be an exceedingly slow, long-term process, occurs on short timescales relevant to ecology, often in response to temporal fluctuations in the environment.  \label{climatescalesfig}}
\end{figure}


\end{document}


``We get only a brief often and dim glimpse of the relevant processes. ... Unless we are aware of the constraints imposed by our scale of operation, and unless we use sound logic in extrapolating from our few frames of observation to a film of ecological and evolutionary processes, we risk either drawing hasty and incorrect generalizations or becoming mesmerized by proximate details that we fail to discern the broader patterns of nature.'' \cite{wiens1986}

%%
%% Seriously, Lizzie, go through this stuff and cut it if you are not using it! 
%%

Conclusions stuff.
% Something about space ... Equivalent advances in temporal ecology have yet to emerge, but have been focus of much recent attention due to anthropogenic climate change.
This is a natural progression:
\begin{enumerate}
\item Data, methods are available now 
\item Theory is coming along (Chessonian coexistence, something else)
\item Development of other fields (Weart 2013 for climate, evolution)
\item Climate change lends some pressing questions, there are lots of good fundamental ones too---people should get on this
\end{enumerate}



\begin{enumerate} 
\item Review: Events, point out it's the messiest definition. They (\(\rightarrow\) trigger regime shifts, state changes etc.)
\item Categories 
\begin{enumerate}
\item Parts of cycles (ENSO years, disturbance etc.)
\item Extreme events (extreme value theory)
\item Black swans
\end{enumerate}
\item Timescales: rapid vs. slow forcing, a few examples (need to pick some of these, not all can go in)
\begin{enumerate}
\item Eco-evolutionary dynamics (Sewall Wright or some other more relevant classic?)
\item Late 1700s drought and North American forests
\item Green Sahara
\item Loblolly example of events versus averaging
\end{enumerate}
\end{enumerate}

\noindent \emph{Time in ecology}\\
Temporal ecology shares several relatively unique attributes with spatial ecology. Like space, the study of time in ecology is fundamentally built on pattern and scale, as we discuss below. \\

Identifying fluctuations and cycles requires also identifying another major temporal attribute of ecological data: noise. Noise is random variation---thus its very definition relies upon statistical or conceptual models which account for other sources of variation, such as cycles, but also possibly trends and events.\\

We argue here that important applied questions, the availability of new data and integration of new methods underscore the need for ecology to reinvigorate its study of how time---a fundamental aspect of ecology---structures populations, communities and ecosystems. 

Much as questions related to habitat fragmentation studies rapidly yielded data at far larger scales than traditionally-available, climate change has brought forward a wealth of long-term datasets in addition to those already available. 



While fluctuations and noise make up two major components of temporal ecology, the importance of time, however, extends beyond to consider the role of all components of temporal patterns, including two features that are critical to examining ecological theory on much longer timescales than traditional (e.g., tens of thousands of years) and to modern processes shaping ecological systems in the Anthropocene: events and trends. \\

Fluctuations and cycles, by their definition, are temporal patterns: fluctuations are variation in some metric over time, while cycles imply more regular variation about some mean. \\

Trends and events represent major components of most time-series data, but have not received extensive ecological study, until recently. Trends represent an important part of nonstationarity. Stationarity---the concept of any process with a probability distribution that does not change when shifted in time---is a major assumption of most time series methods and---we argue---ecological concepts and theory. Stationarity, for example, allows for extensive variation about a mean including possibly cycles and noise, but requires an unchanging underlying distribution such that there are no trends in the mean, cycles or noise. Nonstationarity also allows for the possibility of cycles and noise but requires, in contrast, a shifting probability distribution: such that there are trends in the mean, cycles or noise. Recent climate change, at least partially associated with greenhouse gases (CITE new IPCC), represents an excellent example of nonstationarity. In many systems, climate change has resulted in an obvious trend of increasing mean temperature, with variance about this mean shifted equally (CITE). \\

While the variance may not have shifted, events have.
  
Cycles, alongside events and trends, make up the three major temporal patterns relevant to ecological systems (Figure 2). 

Equivalent advances in temporal ecology have yet to emerge, but have been focus of much recent attention due to anthropogenic climate change. Whilst there has been a rapidly growing body of work on understanding how species and communities will respond to increasing rates of anthropogenic climate change, general theories and paradigms to shape and guide studies have yet to emerge. ``This shortcoming derives at least in part from a lack of integration across fields: on one hand, many
macro[folks] have simply ignored the potential role of biotic
interaction, or barely moved beyond specific examples and broad
consistency arguments, and on the other hand few microevolutionists
and ecologists have gone beyond simple extrapolation from
their hard-won, but short-term, observations'' (Jablonski, 2008). We believe a new framework is needed that parallels efforts in spatial ecology---building on basic theories and concepts---but structured by time (evolutionary process, climatology, time series methods). This temporal focus will require integration across timescales, disciplines and methods. 

% highlight: spotlight, call attention to, point out, single out, focus on, underline, feature, play up, show up, bring out, accentuate, accent, give prominence to, zero in on, stress, emphasize