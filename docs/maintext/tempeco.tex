\documentclass[11pt,a4paper,oneside]{article}
\renewcommand{\baselinestretch}{1.4}
% \renewcommand*{\thefootnote}{\fnsymbol{footnote}}
\usepackage{sectsty,setspace} 
\usepackage[top=1.00in, bottom=1.0in, left=1in, right=1in]{geometry} 
\usepackage{graphicx}
\usepackage{epstopdf}
\usepackage{amsmath,latexsym,amssymb,wasysym}
\usepackage{natbib}
\usepackage{lineno}
\usepackage{todonotes}

\begin{document}
\bibliographystyle{/Users/Lizzie/Documents/EndnoteRelated/Bibtex/styles/amnat}

\noindent Title: \emph{Temporal ecology in the Anthropocene}\\
\\
% Temporal ecology and the future of nonstationary systems
% The importance of temporal ecology to forecasting nonstationary systems
% Building the discipline of temporal ecology
% Temporal ecology and the future of integrative climate change research
% Temporal ecology in a changing world (from Kendra, but I do not like this, I'd prefer 'nonstationary world' if we're doing world or 'shifting' instead of changing
% Temporal ecology in the anthropocene
% Temporal ecology in shifting landscapes/environments
% Temporal ecology: It's about time (from Doug)
\noindent E. M. Wolkovich,$^{1,2,3}$* B. I. Cook$^{4,5}$, K. K. McLauchlan$^{6,7}$ \& T. J. Davies$^{8}$\\
\\
\noindent \emph{$^{1}$Arnold Arboretum, Boston, Massachusetts, United States of America; $^{2}$Organismic \& Evolutionary Biology, Cambridge, Massachusetts, United States of America; $^{3}$Biodiversity Research Centre, University of British Columbia, Vancouver, BC, Canada; $^{4}$NASA Goddard Institute for Space Studies, New York, New York, United States of America; $^{5}$Ocean and Climate  Physics, Lamont-Doherty Earth Observatory, Palisades, New York, United States of America; $^{6}$Department of Geography, Kansas State University, Manhattan, Kansas, United States of America; $^{7}$University of Oxford, Merton College, Oxford, United Kingdom $^{8}$Department of Biology, McGill University, Montreal, Quebec, Canada}\\ %OX1 4JD for Oxford

\noindent *corresponding author: lizzie@oeb.harvard.edu, phone: 1.617.384.5494\\

\noindent Word count (total/allowed): 5761/7500 in main text plus 623 in a box currently, {\bf 199/200 in abstract, 79-80/80 references}, 6 really long figures % Ideas and Perspectives: maximum of 7500 words (main text), 10 figures, tables or boxes, and 80 references.

%%%%%%%%%%%%%%%%%%%
%% Do the Supporting Online Information! %%
%% see http://onlinelibrary.wiley.com/journal/10.1111/(ISSN)1461-0248/homepage/ForAuthors.html %%
%% Number 20 %%
%%%%%%%%%%%%%%%%%%%

% \newpage
\begin{abstract}
Two fundamental axes---space and time---shape ecological systems. Over the last 30 years spatial ecology has developed as an integrative, multidisciplinary science that has helped understand the ecological consequences of habitat fragmentation and loss. We argue that accelerating climate change---the effective manipulation of time by humans---has generated a current need to build an equivalent predictive framework for temporal ecology. Climate change has at once pressed ecologists to understand and predict ecological dynamics in nonstationary environments, while challenging fundamental assumptions of many concepts, models, and approaches. However, similarities between space and time, especially related issues of scaling, provide an outline, for improving ecological models and forecasting of temporal dynamics. While the unique attributes of time, particularly its emphasis on events and its singular directionality, highlight where new approaches are needed. We emphasize how a renewed, interdisciplinary focus on time would coalesce related concepts, help develop new theories and methods, and guide further data collection. The next challenge will be to unite predictive frameworks from spatial and temporal ecology to build robust forecasts of when climate change and habitat degradation will pose the largest threats to species and ecosystems, as well as identifying the best opportunities for conservation. 
\end{abstract}
% The main problem, however, may be that. and highlighted weaknesses in the current framework for temporal ecology
% With increasing availability of long-term records, ecology seems better poised than ever before to tackle such questions, and help work towards an improved framework for temporal ecology that would have large benefits for ecological forecasting.

% Ecologists have long embraced the diversity of temporal issues rampant in much environmental data, and often leverage space to elucidate how temporal processes may drive communities and ecosystems. Complexities in ecological time series and feedbacks inherent in biological systems mean that progress towards understanding how and to what extent systems are driven by temporal patterns require perspectives linking short and long-term dynamics, as well as relevant data. [Slow progress] is understandable given the unique attributes of time, methodological constraints and, until recently, dearth of available time series data. However, climate change, and other aspects of how humans have shaped ecological systems, have highlighted how omnipotent temporal dynamics are. We argue that ecological research would accelerate with a renewed, more constant temporal focus---across and between subfields and related disciplines. 

\newpage
\linenumbers

% \noindent {\bf Introduction}\\
Thirty years ago a transformation in ecological thinking was underway, precipitated in part by questions of how anthropogenic habitat loss and fragmentation affected populations, communities, and ecosystems. Addressing these questions required ecologists to work at scales far larger than their traditional plot sizes, statistical methods and theories allowed, and required integrating perspectives and methods from other disciplines (e.g., geography and evolution) to build upon and develop a body of theories (e.g., island biogeography, metapopulation) and concepts (edge effects and corridors). The field of spatial ecology subsequently emerged from this as an integrative, multidisciplinary science adept at developing concepts and theory to address both basic and applied ecological challenges. Indeed, a major strength of spatial ecology has been its ability to generalize and tackle questions across a broad range of scales, from single-species metapopulations to multi-species metacommunities \citep{Pillai2011} and from local to global scales \citep{bell2001}. As the field has matured, a suite of dedicated journals  (e.g. \emph{Diversity and Distributions, Journal of Biogeography, Landscape Ecology}) has provided forums for the exchange of ideas and cross-pollination between the formerly disparate ideas and disciplines that spatial ecology now encompasses. 

Alongside the human modification of space and rise of spatial ecology, anthropogenic forces have also shifted the temporal dynamics of many systems. This is especially apparent with climate change, which---from arctic to temperate biomes---has fundamentally altered how organisms experience time, and spurred a new body of research aimed at understanding how such temporal shifts impact populations and communities. Much as questions related to habitat fragmentation pressed ecologists to work at larger spatial scales, climate change and related issues have challenged ecologists to better understand temporal processes, over longer time scales. Facilitated in part by improved integration of climate science, evolution and paleobiology into ecology, addressing these questions has rapidly yielded data at larger scales than previously-available. For example, researchers studying phenology have brought together hundreds of thousands of time-series datasets to understand the impact of climate change on the timing of animal and plant life cycle events \citep{Menzel:2006sq}. 

With the increasing availability of long-term data, however, new challanges have arisen. These include creeping timescale issues: population dynamics that appear more complex when examined in longer time series \citep{Ziebarth2010}, selection that weakens when integrated over longer periods \citep{schoener2011,Uyeda2011}, as well as shifts in trends, including responses that reverse over time \citep{yu2010}. Yet a unified field of temporal ecology---with robust theory to explain these issues---has yet to emerge. Instead, within and across disciplines, vocabularies have diverged, often producing different terms for similar concepts (e.g., lag effects \emph{sensu} \cite{Reichmann2013} or carry-over effects, \emph{sensu} \cite{Betini2013}), highlighting the need for a common interdisciplinary forum. 

We argue that there is a compelling current need to develop a unified framework for temporal ecology---one that builds on new data and methods and provides a new focus for predicting how shifting environments shape populations, species, communities and ecosystems. Such a framework could follow the successful, interdisciplinary model of spatial ecology, but would specifically address time. Here we offer a starting point by reviewing the important attributes shared between temporal and spatial ecology, alongside the unique aspects of time that will require new perspectives and methods for robust ecological forecasting. \\

\noindent {\bf Time as a fundamental axis}\\

\noindent  Time is about order and events. In its classical definition it is a dimension that allows (1) sequencing of events from past, present to future and (2) the measurement of durations between these events. Time routinely features in many ecological models and the study of temporal ecology centers on change over time and how such change drives system dynamics. Units can be absolute (minutes, hours, days, months, years) or relative (heart beats, generation times, species’ life spans), and change can take different forms (cycles, trends, noise) and be of different magnitudes, but it is implicit in any ecological process that involves a rate. Ecologists are thus familiar with time as the denominator of many ecological properties, from physiological to community ecology (e.g., metabolic rates, population growth, migration, diversification). Yet time can also shape process, such as species coexistence or predator-prey dynamics. 

Together with space, time represents one of the fundamental axes that shapes ecological systems. In turn these two axes have shaped many of the fundamental questions in ecology. For example, how do spatial and temporal variation in the environment control species' distributions? How does such environmental variation affect population dynamics and structure diverse communities? Such questions highlight that temporal and spatial ecology are intricately intertwined \citep{delcourt1983}, and the two axes share many important similarities.

Similar to space, time---in ecology and across disciplines---is populated by conspicuous patterns. Many time series can be decomposed into regular cycles (e.g., daily, seasonal, multi-annual), longer-term trends and remaining `noise.'  Understanding to what degree fluctuations or cycles in ecological systems are shaped by external temporal patterns or are driven by ergodic properties of populations and species interactions makes up a large portion of study in behavior \citep{macarthur1958}, physiology \citep{Lambers:2008jb}, population  \citep{May1976} and community \citep{Chesson:1997dz} ecology. Trends through time underlie the concept of succession \citep{clementsbook}.  The random variations outside of cycles and trends---temporal `noise'---have also received extensive study by both population \citep{Kaitala1997,Bjornstad:1999kl} and community ecologists \citep{Chesson:1997dz}. Classical community ecology has used temporal variation, including cycles and noise, to explain coexistence via temporal niche partitioning or small-scale differences in species' responses to a temporally-variable environment \citep{macarthur1958,Hutchinson:1961ui,Chesson:1997dz}. Space and time are additionally linked via the importance of scale. Just as spatial patterns may change when examined at local versus regional scales \citep[e.g.,][]{Fridley:2007ct}, temporal trends may appear as cycles, and parts of cycles as singular events, or noise, depending on the timescale. 

Time is unique from space, however, in several important aspects. First, it is impossible to manipulate absolute time. While researchers have manipulated space at small \citep[e.g.,][]{huffaker} and large \citep[e.g.,][]{Terborgh:2001bw} scales, only relative time can be manipulated. Ecologists may adjust the timing of species' interactions \citep{Yang:2010cq}, the sequencing of events \citep{vannette2014}, or underlying drivers of temporal processes to speed up or slow down rates, but they cannot fundamentally alter time itself. Next, temporal patterns are arrow-like---they have each a singular direction. While space may have directional patterns (e.g., altitudinal and latitudinal trends) it is possible to view spatial patterns from almost endless directions and return to a place multiple times. In contrast, time flows. Once an event has unfolded all following patterns and processes may be impacted by it without any temporal recourse to return to it or examine it in another direction. While cycles might give the illusion of returning to a previous point, the temporal landscape has inexorably moved on. Finally, humans experience only a snapshot of time. While ecologists may cover the entire globe to map spatial patterns, temporal patterns over very short or very long timescales are inherently difficult to observe. In the words of \citet{wiens1986},``[w]e get only a brief and often dim glimpse of the relevant processes.'' \\
% Progress towards this goal will require leveraging similarities between spatial and temporal dynamics, while also recognizing unique attributes of time. It will of course additionally require identifying areas where advances could result in the greatest progress towards understanding and onward to prediction. Two major areas that we believe are most critical to rapid advances in predictions are temporal scaling and nonstationarity. 
% Specifically, these new data highlight two of the dominant hurdles that currently hinder accurate predictions: temporal scaling and nonstationarity (Figure 1). Temporal scaling refers to a broad suite of issues, Further, in light of current, largely unprecedented, levels of anthropogenic disturbance, skillful forecasting will also require a consideration of how stationary (i.e., with a constant underlying distribution) biotic and abiotic processes and interactions will be in the future. \\

\noindent {\bf Anthropogenic forcing \& new challenges in temporal ecology}\\

\noindent While ecology has long embraced the importance of temporal dynamics, anthropogenic climate change has posed new challenges. Models of the most basic shifts---in species' ranges or phenologies, for example---are generally built on simple static correlations between ecological and environmental data, and thus may not facilitate predictions beyond the historical record. Further, they have little ability to extend across population, community or ecosystem levels. Part of this shortcoming may be due to chance: in contrast to temporal ecology and climate change, spatial ecology developed alongside a separate and increasing theoretical interest in space---providing ecologists interested in addressing questions related to habitat fragmentation with new theories and models of how space may structure populations and communities. The main problem, however, may be that climate change---by its very nature---challenges fundamental assumptions of many ecological concepts, models, and approaches. 

Climate change introduces into most systems a level of nonstationarity that is largely unprecedented over the last 200 years. In contrast, stationarity---which refers to any stochastic process with a fixed, underlying probability distribution---is a major assumption of most statistical methods and many major concepts and theories in ecology \citep{julio2012}. While all systems are inherently nonstationary at some scale, assumptions of stationarity are often reasonable when the underlying rate of change is slow. For example, while certain environmental factors are still recovering from the last ice age (e.g., rebound of continents following retreat of the ice sheets) and thus nonstationary over long timescales, their trajectory is often so slow that they are effectively stationary when considered against ecological dynamics occurring at shorter timescales. This has changed radically with climate change, which has altered both the magnitude and speed of environmental change in many systems---such that the rate of environmental change now clearly impacts biological systems (Figure 1). More generally, other anthropogenic forcings, including habitat fragmentation and modification (e.g., building of dams), the alteration of disturbance cycles, and the widespread dispersal of invasive and exotic species, can also result in nonstationarity over ecologically relevant time scales.

Improved integration of temporal nonstationarity in ecology requires a more widespread and persistent appreciation of the concept, how it impacts ecological dynamics, and how important it ultimately will be for prediction (Figure 2). Recognizing when nonstationarity is relevant to ecological systems requires addressing issues of temporal scaling, including how processes with differing rates may interact, how species may respond to the same forcing over different time intervals (e.g., daily versus annual versus interannual temperature fluctuations; changes in extreme events versus the mean), and the appropriate temporal span and sampling frequency required to draw conclusions regarding trends, variability, and periodicities \citep[e.g.,][]{delcourt1983}. This will require revisiting basic ecological paradigms in a new light and adapting relevant theories and models to incorporate temporal nonstationarity. \\

\noindent \emph{Nonstationarity in current ecological models}\\
\noindent Temporal nonstationarity is not a new concept in ecology. Many of ecology's major concepts are descriptions of temporal nonstationarity, including much work focused on disturbance (e.g., the shifting mosaic hypothesis), regime shifts and alternative stable states, as well as extinctions and extirpations. Ecology, however, has an uneven history of embracing temporal nonstationarity and considering it in both drivers of ecological systems and in ecological responses. This is perhaps best illustrated by changing views on the concept of succession (changes in the structure and function of ecosystems over time) and its relationship to the abiotic environment (Figure 3). Successional theories underlie one of the classic debates in ecology, pitting Clementsian versus Gleasonian versions of nature against one another \citep{clementsbook,gleason1926}. In the Clementsian version, communities shift over time in a predictable sequence that is not highly impacted by the abiotic environment (Figure 3a). A clear example of this view comes from studies of primary succession on newly-deglaciated surfaces in Glacier Bay, Alaska that describe a temporally-predictable sequence of four vegetation types: pioneer, Dryas, alder, and old-growth spruce and hemlock forests \citep{cooper1923}. Over this sequence, ecosystem properties changed over time \citep{Chapin1994}, with the rate (but not the endpoint) of succession controlled by biotic interactions and a minimal or nonexistent role for the abiotic environment (e.g., climate). \citet{gleason1926} offered an alternative view of succession, stressing the importance of the abiotic environment and, thus, expected far less predictable successional trajectories. This later view recognizes that events such as climate extremes and other disturbances could reset successional clocks (Figure 3b), and thus produce diverse ecological patterns across the landscape \citep{Levin:1992rg,romme2011}. While succession is fundamantally about temporal nonstationarity in an ecological process, it is not, however, fully developed to handle temporal nonstationarity in underlying drivers. Rapid shifts in climate, for example, could shift trajectories or make it impossible for systems to return to a given trajectory following a disturbance (3c). In this way, temporal nonstationarity may be a key predictor of regime shifts in communities and ecosystems.
% Related to this, concepts of how systems respond to events, including definitions of resilience \citep{Grimm:1997} and multiple stable states \citep{yang2010} are fundamentally about how nonstationary systems are over the time period of interest. 

A framework for better incorporating nonstationarity into ecological models will require consideration of both nonstationarity in the forcings (e.g., climate, Figure 3a) and also in the ecological responses (Figure 4b1, b2). Nonstationarity in climate may push species outside of their normal response range. For example, many species will advance their phenology with warming in a linear fashion until a certain threshold, after which phenology may be dominantly controlled by photoperiod or snow cover \citep{Iler2013}, resulting in nonstationarity in species' responses to climate change. Many current ecological models could be adapted to make predictions with climate change if stationarity assumptions were relaxed. This could include adjusting population ecology models to examine outcomes when life history parameters related to the environment (e.g., mortality due to drought etc.) are modeled as nonstationary. In community ecology, coexistence models built on temporal variability \citep[e.g.,][]{Chesson:1997dz} are prime candidates to examine the consequences of environmental nonstationarity on community structure. 

Ecology also must become more aware of stationarity assumptions in its statistical methods. General linear models can often be adapted to include temporal autoregression; this may accommodate some temporal nonstationarity, but could equally hide its impact. A better approach would be to explicitly model temporal nonstationarity, which will often require new model development and further integration of models from other disciplines that allow nonstationarity \citep[e.g,][]{Grenfell:2001ox,lipp2002}. Such models are often also used in spatial ecology, with its increasing recognition of nonstationarity across space, and have led to new hypotheses and methods. For example, geographically-weighted regression relaxes the assumption that process parameters (e.g., variance and mean) are independent of location and direction \citep{Brunsdon:1998nx}, allowing researchers to test questions of whether differing processes---which shift across space---could shape fundamental ecological patterns, such as species richness \citep{davies2011}.\\  %  In temporal ecology, implicit assumptions of nonstationarity pervade certain arenas and underlie some of the discipline's oldest concepts, and debates. Yet ecological forecasting will require ecologists to more fully embrace how ubiquitous and important nonstationarity is to both abiotic drivers and biotic responses. % The concept of temporal nonstationarity is particularly critical for forecasting ...

\noindent \emph{Temporal scaling}\\ 
\noindent Scaling issues in temporal ecology mirror many of the same challenges in spatial ecology, including the grain and extent of sampling. For example, while spatial ecology is concerned with how well observations at the plot level ($10^{1}m^2$) scale to the landscape level ($10^{2}-10^{3}m^2$), temporal ecology must harmonize across ecological processes that span disparate temporal extents and observation at various temporal grains, from minutes (e.g., photosynthesis) to days and weeks (e.g., phenology, annual productivity) and upward to centuries and millennia (e.g., successional dynamics, carbon cycle dynamics, evolution of species' niches). Issues with temporal dynamics observed at short timescales that do not appear to impact long-term dynamics have fueled many of ecology's most vociferous debates \citep{wiens1986}. Mismatches in temporal grain are highlighted by the difficulties inherent in incorporating fast and slow processes in models of ecosystem dynamics \citep{Carpenter2000} or modeling temporal community coexistence via both seasonal (e.g., phenology) and interannual environmental dynamics \citep{Chesson:1997dz}. Climate change has refocused ecological thinking on temporal scaling, providing a major new impetus to revisit fundamental questions and identify where scaling issues limit predictions. Particularly critical for understanding ecological responses to climate change is bridging from the shorter, more rapid temporal scales that characterise ecological responses to the longer evolutionary timescales that have shaped species' responses and to the slower processes such as rock weathering, ecological succession, and some climate system dynamics (Figure 5). 

Studying too short a timescale (narrow extent) can mean that important long-term cycles or slow processes are missed, which can hamper prediction. Many annual population, community and ecosystem dynamics are at least partly driven by multi-annual climate cycles. For example, the highest anomaly of global net primary productivity (observed in 2011) was attributable to high precipitation due to the strongest La Ni\~na year recorded \citep{bastos2013}. Thus, consideration of the El Ni\~no Southern Oscillation (ENSO)---a climate feature with approximately a five-year periodicity---is essential in many ecological systems. Failure to incorporate ENSO dynamics when considering water budgets in the western U.S. has led to persistent problems with sustainable water supply under the Colorado River Compact \citep{Christensen2004}. 

Conversely, observations collected at too large a temporal scale (coarse grain) rarely scale down to shorter timescales. For example, the temperature sensitivity of ecosystem respiration derived from annual datasets does not reflect the short-term temperature sensitivity calculated by extrapolating from night to daytime data \citep{Reichstein2005}. For climate change responses, basal physiological timescales---such as daily metabolic or photosynthetic rates---are often more important for understanding ecological responses to the environment. Photosynthesis, for example, responds to daily variations in temperature and light availability, which then integrates to gross and net primary productivity that will be additionally limited by weekly to monthly climate and weather variability (e.g., heat waves, frost events).  In an additional example from plants, phenology responds to climate at the daily timescale (daily heat accumulations, snowmelt, first rains of the wet season, photoperiod, etc.), but is critical for defining the beginning and end of the growing season, setting fundamental limits on biological activity any given year. Yet, despite the recognized importance of these basal scales, many ecological models have historically used climate data available at the monthly scale \citep{Sitch2003}, leading to a disconnect between the timescale of ecological theory and the temporal resolution of the data.
% Ref for photosynthesis above (from Ben): Can we just use this textbook ref for the photosynthesis example in the scaling section (Line 153-156)? http://www.springer.com/life+sciences/plant+sciences/book/978-0-387-78340-6

This issue of temporal grain is perhaps most clearly illustrated by climate envelope models that are often used to predict species distributions under climate change. Such models frequently use seasonal and annual average temperatures as the primary constraints on species ranges, but much evidence indicates that ecological processes---including species' ranges---are limited not by mean climate, but rather the recurrence intervals of extreme events (e.g., fire, droughts) or higher order climatic moments (e.g., coldest winter day). For example, the distribution and population sizes of many insect pathogens are limited not by average summer or annual temperature controls on fecundity, but by mortality induced by minimum winter temperatures \citep{weed2013}. Further, increasing evidence from the paleorecord indicates that extant species have occupied quite different climate conditions in the past: in another insect example, beetles that have survived several glacial/interglacial cycles in Europe appear to have had significantly different ranges---and thus survived in relatedy different environmental conditions---throughout the Quaternary \citep{Abellan2011}. These observations suggests that at the very least, a more explicit consideration of time might lead to caution in the use of species distribution models under conditions very different than modern. 

% EMW: I adjusted from here on to the next section, some cutting, some moving a adding back in a little cut text, see what you think.
Related downscaling issues can be seen in attempts in evolutionary biology to link short timescales on which ecological dynamics often occur to the longer timescales that shape species and lineages. For example, \citet{lavergne2013} related rates of evolution derived from phylogenetic trees to species' current sensitivities to recent environmental change; this tacitly assumes information from two very different temporal scales---millions of years versus decades to centuries---can be simply and directly linked. Timescale issues have also hampered efforts to estimate evolutionary rates \citep{schoener2011,Uyeda2011}. Over short timescales observations suggest rapid, but bounded evolution, consistent with population divergence over a fluctuating adaptive landscape within an adaptive zone
\citep{Uyeda2011}. While, over the longer time-scales, sufficient for speciation, variance increases slowly, but more or less linearly---consistent with rare niche shifts that reconfigure the adaptive landscape. Reconciling this apparent disjunct seems a primary goal before phylogenetic information could be robustly incorporated into ecological models of species' responses to climate change.

Several basic approaches in ecology can help to identify and reconcile temporal scaling issues; in particular complementary timescales provided by differing approaches can be leveraged to address the same question. Ecological approaches have often been abstracted into experiments, observations, long-term observations and modeling \citep{Carpenter:1992hk}. Experiments are often conducted on the shortest timescales---from days to weeks in the lab, to weeks and years in the field---and may only capture transient dynamics. Experiments generally allow, however, the most powerful tests of mechanisms and, relatedly, major insights \citep{Chapin:1995af,Wolkovich:2012n,Laube2014}. Such tests are buoyed by comparisons with observational data, both short-term (e.g., a single growing season or several years) and long-term. Modeling can help to understand dynamics and generate predictions beyond the scale of observations---possibly by incorporating important longer-term dynamics such as climate cycles---and ideally provide new insights and predictions to test. Today, ecology has a significant advantage in integrating across timescales through increased interdisciplinary work with other fields, especially climate science, paleobiology, and evolution.

For studies focused at a single temporal scale the key is to match the timescale of observation with the timescale of the process. Just as landscape ecology required identification of the relevant spatial scale for sampling, temporal ecology requires identification of the relevant temporal grain and extent for addressing the ecological question of interest. As a first step this means recognizing the relevant timescales---including the generation times of the study organisms, frequency of disturbance, as well as the period of climate oscillations---and then placing the study in the relevant part of these cycles. Improved integration of temporal scaling in ecological studies, however, will require continually cross-checking approaches of varying timescales, modeling studies to extend beyond currently available data, and more integration of disciplines that have sometimes worked separately because of their underlying disparate timescales (Figure 5). \\

\noindent \emph{Events: Where scaling and nonstationarity intersect} [Better title suggestions?]\\
Rapid bouts of evolution that reshape niches, extreme frosts that limit species' ranges and disturbances that alter community trajectories all highlight a major feature of temporal ecology: events. Events---the presence in time series data of noncyclical and/or abrupt, and often nonstationary temporal patterns---are one area where temporal ecology diverges from spatial ecology. While the term `event' has taken multiple meanings in the literature (see Box: Defining Events), within ecology, it typically refers to single, discrete occurrences, such as seed dispersal \citep{Higgins2003}, establishment of a new species \citep{Blackburn2011}, an extreme forcing such as a frost or drought \citep{Jentsch:2009ff} or a much larger climate shift, like the Younger Dryas \citep{Jackson:2009el}. Here, we view events more holistically, and recognize that forcing events may be discrete (e.g., a frost or drought) or continuous (e.g., long-term warming trends), and the ultimate ecological responses may be transient (temporary) or continuous (persistent). Importantly, any ecological response to a forcing will depend on the system dynamics, including feedbacks (positive or negative), and resistance or resilience inherent to the system of interest. 

Events highlight temporal scaling issues as detection depends on three sampling features: the temporal frequency of sampling (grain), the duration of the sample period (extent), and the magnitude of the event or departure from previous samples. Events may not be detected or perceived as events if any of these three features is insufficient (Figure 6). For example, sampling of plant species that flower at the beginning of a temperate growing season lead to the conclusion that the year 2012 was three standard deviations beyond the mean, and therefore could be considered a transient response event. However, additional sampling of plant species that flowered later in the growing season demonstrates a less extreme response. Thus, the characterization of the year 2012 as an event that affected plant phenology would be dependent on sampling and the time period of interest. Similarly, a relatively cold interval in the middle of a warming trend could be either a late spring frost (one day), or the Younger Dryas climate event \citep{Jackson:2009el} depending on the duration of the event relative to the duration of the sampling period. 

Events may also be characterized by significant nonstationarities in ecological systems. Examples include regime shifts in aquatic communities caused by food web shifts \citep{Carpenter2011}, irreversible ecosystem changes caused by disturbance events \citep{Allen1998}, or changes in ecosystem structure and function with the removal of key species \citep{mumby2007}. Because many extreme forcing events (e.g., droughts, heat waves) and their consequences (e.g., community shifts, extinctions) are often rare, predicting their occurrence and ecological impacts has been difficult. Recent efforts, however, to test models of regime shifts \citep{Carpenter2011,Boettiger2013} and to predict the outomes of contigency events in shaping communities \citep{vannette2014} have shown promise. Further efforts, however, may be compounded in the context of climate change, where events occur in a steadily shifting baseline climate (Figure 5). \\

\noindent \emph{Forecasting ecological systems in nonstationary environments}\\
\noindent Current pressing questions in temporal ecology would advance our basic understanding of ecological dynamics. These include: when can we extend inferences from one time period or timescale to another? How and when does temporal nonstationarity shape ecological dynamics and system trajectories? What are the attributes of events and temporal nonstationarity that create persistent shifts in ecological systems? What are the conditions or timescales when abiotic versus biotic drivers tend to dominate ecological dynamics?  

Addressing these questions would also, however, make fundamental contributions to expanding and improving predictions in ecology. They would critically help answer whether inferences drawn from contemporary and historical data are appropriate for a future world with quantitatively different boundary conditions. For forecasting, researchers must also address scaling issues with the often coarser (i.e., larger grain) temporal data available for model calibration and prediction. For example, understanding how a species responds to climate change must consider how a species' response to a persistent increase in mean temperatures over many years may differ from the much larger---but shorter-term---fluctuations that many populations and species experience on a daily or weekly basis (Figure 5), and whether responses across such timescales are linked. Relatedly, given that most species ages are 1-10 million years \citep{lawtonbook} the best projections would also consider how a species has responded to previous major climatic shifts, which are often equal in magnitude and rate to current climate change (Figure 5).

Projecting shifts in communities with nonstationarity should benefit from increasing recognition of how temporal nonstationarity structures ecological communities. For example, research on historical contingencies and temporal legacies may help forecast communities in nonstationary systems. Studies of community and ecosystem stability \citep{Boettiger2013}, paleoecological systems and modern disturbance ecology have provided foundational work on the role of contingency in driving ecological systems and highlighted that historical contingency is often more common than predictable, deterministic sequences over time. Moving forward, the challenge is now to build theory that incorporates contingency and devleop studies that can provide more robust tests of how contingencies operate \citep{vannette2014}. More research is also needed on the role of multiple or compound disturbances in altering trajectories and on how environmental nonstationarity may make regime shifts more common, by effectively moving the underlying environmental track (Figure 3c). % The importance of historical contingency in ecology should not in any way suggest that systems are less predictable, it only highlights what ecologists already know so well---that ecological systems are complicated and accurate prediction requires careful knowledge of many driving factors.

Given the importance of nonstationarity, and the increasing evidence that ecological trajectories are often not deterministic, ecological forecasting may additionally benefit from probabilistic, rather than deterministic, modeling approaches, such as those used in the field of climate modeling \citep{Tebaldi2007}. A probabilistic approach allows for better understanding of the internal, unpredictable variability in the system. For example, many climate model projections use an ensemble approach, where individual models (i.e., climate projections) start from different initial conditions but use the same set of forcings and boundary conditions (e.g., land cover, greenhouse gas concentrations, etc.). In this way a spread of projections is generated, with probabilities of future climate states emerging naturally as a function of the forced response (ensemble average) and internal variability (ensemble spread). Probabilistic sampling and modeling may also allow for detection and attribution of controversial topics in ecology for which data are limited, such as $CO_{2}$ fertilization and invasive species, as well as understanding the importance of very rare, but random events, such as Black Swans (Box 1).\\

\noindent {\bf Combining the axes of space \& time}\\

\noindent A more robust framework for temporal ecology will allow uniting the predictive--and intertwined---frameworks of spatial and temporal ecology. We argue that advances in temporal ecology could be motivated spatial ecology's emphasis in recognizing and understanding hidden dimensions in ecological models and theory. While in turn, decades of progress in understanding the consequences of spatial processes have resulted in a return to the importance of temporal dimensions in ecology. For example, island biogeography theory predicts species richness based on several basic spatial metrics---but temporal dimensions of the controlling processes---immigration, extinction and speciation---are also fundamental to predictions \citep{Wiens2011}. Similarly, disease models have advanced through incorporating both spatial and temporal models of traveling waves as disease prevalence varies both with population density and temporal fluctuations in that density \citep{Grenfell:2001ox} and climate \citep{lipp2002}. In paleoecology, research has advanced to visualize past vegetation assemblages in both space and time by combining data from across diverse sites and spanning 10,000 years \citep{Brewer2012}. Perhaps the current best example of spacetime integration comes from outside of ecology, and instead from climatology, where methods, such as empirical orthogonal function analysis, explore at once temporal and spatial patterns in climate data, and could be employed in examining some of the longer-term, spatially-explicit datasets in ecology.
% cut example: At the ecosystem-scale, spatial and temporal patterns of ecosystem carbon flux are being studied using a network of 253 eddy covariance research sites in the FLUXNET global database. With nearly 1000 site-years of data, low frequency and high frequency signals in the climate signal are being identified so that the net effect of climate change on ecosystem carbon exchange can be isolated [Stoy et al. 2009 KM]. 
% For example, island biogeography can help make predictions of species richness, but assumes stationarity in the mainland population even though few mainland populations in recent times are stationary. 

These recent advances represent, however, only a small foray into the potential benefits possible from fully embracing the interconnectedness of spatial and temporal dynamics in ecology. Consider coexistence theory---long stymied by models that required \(n\) different axes to produce \(n\) coexisting species alongside empirical examples of many co-occurring species that appeared quite similar when examined from one snapshot---it advanced when the role of variability in species' responses to the temporal dimension was re-examined \citep{Chesson:1997dz}. Under the storage effect model, highly similar species coexist via small differences in how they respond to temporal variability in the environment. Since its introduction the storage effect model has been ported to spatial dimensions---where species coexist via reduced competition from spatial variability. Tests for such models have found support separately for temporal \citep{Angert:2009} and spatial \citep{Sears:2007md} storage effects, but we expect most communities function based on a constantly shifting mix of the two mechanisms. For example, studies of community change during the Dust Bowl show how dominant species may decline to apparent local extinction while rare species rise to abundance \citep{Weaver1936}. Recent work highlights that the persistence of rare species throughout long time periods may come from landscape dynamics where microclimates maintain high species diversity \citep{Craine2012}. In such cases temporal storage effects are built on buffered population growth maintained by spatial dynamics. Further, by modeling the environment explicitly, such models could make predictions of how fundamental coexistence mechanisms may shift with climate change and help answer critical questions of how communities built on coexistence mechanisms via a temporally and/or spatially variable environment will respond when that environment switches from stationary to nonstationary. 

Finally, robust projections of climate change impacts on populations and species will require an adjustment to the most classic spatial metaphor for a temporal process: adaptive landscapes. Nonstationarity in climate has resulted in rapid and effectively continuous shifts to most populations' adaptive peaks and valleys. Climate change has thus highlighted how rapid evolution may be and has brought it firmly into an ecological timescale, but theory as to how such nonstationarity may affect evolutionary outcomes remains a challenge \citep{schoener2011,Bailey2014}. \\ % THIS HIGHLIGHTS JUST A TASTE OF WHERE WE COULD GO. %Given the tremendous diversity of temporal variation in most environments it should not be surprising that it would be a major axis on which to mute competitive forces and thus promote coexistence.

\noindent \emph{Spacetime in conservation ecology}\\
\noindent Over thirty years ago, spatial ecology emerged as a unifying framework for analyzing and interpreting spatial patterns across disparate disciplines, motivated by increasing rates of habitat transformation and fragmentation. Today, a similar structure is developing around the concept of time, introduced here as temporal ecology, and driven primarily by the need to better understand ecological responses to climate change.  While habitat loss has been the main driver of extinctions historically, climate change poses perhaps the biggest threat to biodiversity in the future, yet we lack equivalent general theories and paradigms to shape and guide research efforts. We believe a new temporally-focused framework that integrates across timescales, disciplines, and methodologies, and borrows from the fields of evolution, ecology, and climate science would help in developing methods, concepts and theories. Recent advances within subfields incorporating environmental variability into coexistence models \citep{Chesson:1997dz}, bridging ecological and evolutionary timescales \citep{schoener2011}, revisiting the role of climatic events in setting range limits \citep{Tran2007} and in modernizing paleoecology \citep{Brewer2012} indicate that the discipline of ecology is up to the challenge. % Whoo! Ben is so good at being positive.

A renewed temporal ecology framework has particular relevance for conservation science in the Anthropocene---where ecological dynamics operate in increasingly nonstationary environments dominated by rising rates of anthropogenic change. Traditionally, conservation biology has focused on space---identifying the best locations to conserve species or habitats \citep[e.g.,][]{Cincotta:2000gk}, motivating the establishment of reserves, refugia, and corridors \citep[e.g.,][]{Doak:1989oc}. But while the ecological consequences of habitat loss may be more obvious or immediate, nonstationarity in climate highlights the necessity of also considering changes over time \citep{Hannah2002}. For example, species range shifts associated with climate change suggest that policies for setting conservation areas must not only consider current suitable areas, but also how these areas might change in the future. Within a climate change scenario, the very concept of conserving biodiversity within fixed protected areas may be misguided \citep{Rutherford1999}. A joint consideration of space and time may help resolve some of the current debates on tradeoffs between prioritizing species conservation for habitat loss (space) and climate change (time), and a dual consideration of both space and time will allow the identification of where and when the best opportunities exist for mitigation and conservation.

A broader temporal ecology perspective may also help inform the probability and potential impact of extreme events, such as Black Swans, and the resistance and resilience of ecosystems to these events. For example, a species or ecosystem may adapt to long term changes in the average climate (e.g., long term warming), but recent events in many landscapes \citep[e.g.,][]{Anderegg2013} highlight that the frequency and impact of extreme events (such as drought and insect irruptions) may fundamentally alter ecological responses. Conservation strategies must therefore consider the impact of such events and how they may impact the resistance and recovery of ecosystems to further events in the future. Insights into these issues can be gained from historical and paleoecological data but projecting into the future will require recognizing the nonstationary nature of these processes. Conservation would thus benefit from both spatial and temporal considerations that may require, for example, establishment of protected area networks that encompass diverse topographic landscapes and include migration corridors, which allow species to better adapt to extreme events in the future. \\

\noindent {\bf Conclusions}\\

\noindent The two greatest threats to ecological systems in the Anthropocene---habitat degradation and climate change---represent human modification of space and time, shifting the fundamental axes of ecological systems. As ecology is challenged to better understand and predict these changes gaps in our body of concepts, theories and methods have appeared. Such gaps, however, also highlight opportunities for advances in both basic and applied ecology. In the twentieth century, classical Newtownian physics gave way to Einstein's theory of relativity with the recognition that time is not simply a fourth dimension orthogonal to space, but a relative metric, inherently intertwined with space: Ecology now has an opportunity to build a more complex spatiotemporal perspective. Clearly, ecology has progressed significantly in recent decades as ecological data spanning years, decades, and centuries have become increasingly available in paleo-, conservation, community, and ecosystem ecology. The challenge remains, however, to develop a holistic structure that will allow for cross-disciplinary sharing of methods and ideas and leverage the strengths of these disparate fields. Encouragingly, such work is being developed in areas like phenology \citep{Pau:2011wd}, paleoecology \citep{Brewer2012}, and conservation \citep{mooers2008}, suggesting there is great potential for rapid advances. \\

\noindent {\bf Acknowledgments}\\
We thank D. Bolger, M. O'Connor, and D. Schluter for comments and conversations, and M. Donohue and S. Brewer for conversations, that improved this manuscript. EMW was supported by the NSERC CREATE training program in biodiversity research.  % We thank both Monsters and Men. 


\newpage
\noindent {\bf Defining events}\\ % ADD Ben's edits (copy_BIC)
\noindent Time is fundamentally about events, with research often aimed at quantifying their occurrence, duration, and sequencing. `Event' is thus commonly used in any temporally-focused literature, including global change ecology. Despite its ubiquity, however, a precise definition has eluded the term in the ecological literature. For example, `event' has been applied to a variety of biotic and abiotic phenomena, including fire, establishment of invasive species, drought, insect irruptions, frosts, and others. Improved understanding of temporal events in ecological systems would thus benefit from a more nuanced approach, with clear and precise language. One area of temporal dynamics that is of particular interest is how quickly and persistently ecosystems respond to forcings (see Figure) that may be either short-lived, \emph{transient} events (e.g. fires, droughts, insect defoliation, etc.) [1] or persistent, \emph{continuous} changes in the background state (e.g., climate change, introduction of invasive species, habitat fragmentation, etc.) [2]. Similarly, however, these continuous forces may give rise to transient ecological responses [3] or continuing responses [4]. The permanence and velocity of ecological responses depends not only on the nature of the forcing (e.g., its severity and duration), but also on the inherent capacity for resistance, resilience, and feedbacks within the ecosystem or community of interest. Thus events may be better specified in terms of whether they are related to the forcing or response, and whether they are transient or continuous.\\

\begin{figure}[h!]
\centering
\noindent \includegraphics[width=0.4\textwidth]{/Users/Lizzie/Documents/git/manuscripts/temporalecology/figures/events/events.png}
% \caption{}
\end{figure}

For example, vegetation may quickly return to its previous state following transient disturbances, such as a fast growing grassland recovering after a fire or drought \citep[e.g.,][]{Weaver1936,albertson1944}, or a plant down-regulating initial photosynthetic enhancement in response to elevated $CO_{2}$ concentrations \citep{eatonrye2012}. Both responses can be considered transient, regardless of the nature of the forcing, and may indicate either some inherent resilience in ecosystem structure and function (in the grassland example), or fundamental shifts in the in the importance of the resource limitation and environmental stressor space. Ecosystems may also respond in persistent ways to either transient or continuous forcings. A relatively recent example is the switch from a ponderosa pine forest to a pi\~non-juniper woodland in southwest North America following a major drought in the 1950s \citep{Allen1998}. This new woodland persists to this day, despite a subsequent return to more normal moisture conditions. In another example, during the Mid-Holocene the Sahara permanently shifted from a woodland savanna to a hyper-arid desert in response to changes in Northern Hemisphere summer insolation, with the ecosystem collapse happening much more quickly than the forcing change \citep{Foley2003}. Clearly, the nature of forcing events (fast or slow, discrete or continuous) does not necessarily map clearly onto ecological responses, presenting a challenge for better prediction of the speed and persistence of ecosystem responses.

Additional difficulties may be presented by a special class of events known as `Black Swans'. A Black Swan event is defined by two components: (1) that it has dramatic effects on the system, but is extremely rare, such that (2) it is effectively impossible to predict using current methods. These two components lead to the third aspect of Black Swan theory: owing to their large impact on the system there is a strong tendency to believe such events can be predicted---when, instead, by their extreme rarity this is generally impossible.  There is already evidence for ecologically important `Black Swan' events. One example is an 18th century drought in eastern North America that has shaped successional trajectories to this day \citep{Pederson2014}. While another, more well known example, is the Salton Sea, an inland body of water in southern California that formed during a large flood event in the early 20th century, and subsequently became a critical habitat for wildlife and migratory birds \citep{Cohn2000}. Identifying these events and their importance for ecological processes in historical and paleoecological data, however, remains challenging.\\ 

% (Moved up currently.) To achieve the goal of better understanding the importance and time scale of forcings and ecological responses, ecological modeling may benefit increasingly from probabilistic, rather than deterministic, modeling approaches, such as those used in the field of climate modeling (ref XXXX). Using a probabilistic approach allows for better elucidation of the internal, unpredictable variability in the system and the full exploration of the possible model parameter space. For example, many climate model projections use an ensemble approach, where individual ensemble members (i.e., climate projections) start from different initial conditions but all use the same set of forcings and boundary conditions (e.g., land cover, greenhouse gas concentrations, etc). In this way a spread of projections is generated, with probabilities of future climate states emerging naturally from the ensemble as a function of the forced response (ensemble average) and internal variability (ensemble spread). Probabilistic sampling and modeling may also allow for detection and attribution of controversial topics in ecology for which data are limited, such as CO2 fertilization (ref XXX) and invasive species (ref XXXX), as well as understanding the importance of very rare, but random events, like Black Swans.


\newpage
\begin{footnotesize}
{\def\section*#1{}
\bibliography{/Users/Lizzie/Documents/EndnoteRelated/Bibtex/LizzieMainMinimal}
}
\end{footnotesize}

\newpage


\begin{figure}[h!]
\centering
\noindent \includegraphics[width=0.9\textwidth]{/Users/Lizzie/Documents/git/manuscripts/temporalecology/figures/nonstationarity/swissgrapes.png}
\caption{Temporal ecology is focused on understanding how, when and where a temporal framework is needed to understand ecological systems; this importantly includes examining when drivers and responses are stationary versus nonstationary. Nonstationarity in temporal data occurs when the underlying probability distribution shifts across time. Until recently many systems appeared stationary over the timescale of most recorded ecological data (i.e., within the last 100-200 years), as is seen here in grape harvest records from Switzerland \citep{Meier:2007zh}. Yet systems may appear nonstationary when examined during other periods (e.g., the 1700s in these data). Importantly, many systems now appear nonstationary with increasing rates of climate change. For many systems, climate change has resulted in an obvious trend of increasing mean temperature, with variance about this mean often shifted equally \citep{Huntingford2013} (though the concept of nonstationarity can result in shifts beyond the mean---including shifts in cycles or noise). Shifts in mean temperatures may in turn impact biological processes, for example by advancing phenological events such as harvest---as seen in the last several decades of this record. See Online Supporting Information for details on data.}
\end{figure}
% should change from Meier ref to data ref they ask to have used! Also add info to supplement.

\begin{figure}[h!]
\centering
\noindent \includegraphics[width=0.5\textwidth]{/Users/Lizzie/Documents/git/manuscripts/temporalecology/figures/prediction/prediction.png}
\caption{Long-term records in ecology such as repeated measures or observational data spanning at least 5 to 10 years are increasingly common. Such data provide an opportunity to improve understanding and prediction, but also a challenge of how best to interpret trends in such data. Depending on the system and period of observation, what looks like a linear increase (a) could be part of a regular long-term cycles (b), indicative of a major shift in the system into a nonstationary period (c) or possibly part of both (d), especially if forcing on the system has changed---as seen in many systems with climate change. Temporal scaling and nonstationarity are, thus, inherently linked as any system or process can look stationary or nonstationary depending on the scale.}
\end{figure}

\newpage
\begin{figure}[h!]
\centering
\noindent \includegraphics[width=0.5\textwidth]{/Users/Lizzie/Documents/git/manuscripts/temporalecology/figures/assembly/assembly.png}
\caption{The evolution of how ecologists view time and the role of the environment in shaping temporal dynamics can be seen partly in the maturation of theory on succession---a fundamentally nonstationary ecological process. Early work (a) tended to focus on one possible trajectory and outcome. A consistent, predictable turnover of species was the main driver of ecosystem development; mean climatic factors shaped the species pool on the largescale, but climate was otherwise generally unimportant. As work progressed (b), ecologists recognized that multiple trajectories were possible---often triggered by climate extremes and other related disturbances (e.g., drought, fire) that reset the relative temporal position of an ecosystem along its development curve. More recently, ecologists have layered onto this an appreciation of factors that may yield diverse trajectories and endpoints. Additionally, research on tipping points and alternative stable states have highlighted that some events may transition ecosystems to fundamentally different states; nonstationarity in climate, or other ecosystem drivers, may contribute to such tipping points as it could be difficult for systems to return to trajectories if the underlying climate has shifted significantly while the system recovers from the disturbance (c).}
\end{figure}

\newpage
\begin{figure}[h!]
\centering
\noindent \includegraphics[width=0.8\textwidth]{/Users/Lizzie/Documents/git/manuscripts/temporalecology/figures/nonstationaritywphen/nonstationarity.png}
\caption{Forecasting ecological responses to climate change requires layering projections of complex physiological responses onto nonstationary drivers, such as increasing temperatures (a), shown with a 10-year locally weighted scatterplot smoothing in red. Much research has focused on projecting phenological responses to this increasing temperature (b), which requires understanding whether responses are fundamentally linear (b.1), where higher spring temperatures yield earlier leafing or flowering or non-linear (b.2) where responses to temperature are limited at higher values by additional factors, for example photoperiod, though drought, nutrients or other factors could also be critical. Such responses may look identical under a stationary climate regime (lighter blue shading), but would become apparent once the threshold is crossed under nonstationary climate (darker blue shading). See Online Supporting Information for details on data in (a).}
\end{figure}

\newpage
\begin{figure}[h!]
\centering
\noindent \includegraphics[width=1\textwidth]{/Users/Lizzie/Documents/git/manuscripts/temporalecology/figures/ecoevoclimate/climateEcoEvo_alltogethernowPlus.png}
\caption{Robust forecasting in temporal ecology requires recognizing the multiplicative dimensions of time inherent in most ecological processes (top arrows). For example, predictions of species' responses to climate change must consider: (1) that many species experience far larger shifts in temperature on the timescale of hours to days (left) and (2) that over their evolutionary history many species have experienced climate swings similar in magnitude and rate to current and projected anthropogenic climate change (middle), in addition to the pressures of glaciation cycles (right). Differing methods in ecology (bottom) are optimized to differing timescales but ecologists are generally most adept at working in timescales of days to years. See Table S1 of the Online Supporting Information for details on data and references.}
\end{figure}


\newpage
\begin{figure}[h!]
\centering
\noindent \includegraphics[width=0.8\textwidth]{/Users/Lizzie/Documents/git/manuscripts/temporalecology/figures/extremes/extremes.png}
\caption{Paralleling grain and extent in spatial ecology, in temporal ecology, both the temporal frequency (a) and duration (b) of sampling are important. This can be seen when considering the utility of time series data to understand how climate impacts ecological dynamics. In plant phenology research (a), within-year sampling is often focused on the start of spring and a small number of species, with fewer observations of species that flower later in the season. Increasing the frequency of sampling such that more species are studied across the growing season can affect the interpretation of a climate event. For example, 2012 was an extremely warm spring for much of northeastern North America and early season species, such as \emph{Prunus serotina}, flowered much earlier than ever previously observed ($z$-scored data show 2012 was three standard deviations beyond the mean for this species, inset shows raw time series data), while many other species, such as \emph{Betula lenta} that flower several weeks later, did not show as extreme a response. Duration of sampling can also easily affect how extreme an event---such as a drought (b)---appears. Considering only the 20th century (b, see light blue bars in histogram, and light blue shaded area of inset showing the Palmer Drought Severity Index, where positive values indicate wetter than normal conditions and negative values indicate drier than normal conditions), a 4-year drought in central North America appears uniquely long. Extending the record, however, to the last 1,000 years (gray bars in histogram) shows far longer droughts throughout the 12th to 15th centuries (arrows give years of droughts). See Online Supporting Information for details on data.}
\end{figure}

\end{document}

\noindent \emph{Text parking lot:}\\
It has provided a major new impetus to revisit fundamental questions whose answers have applications far beyond climate change: for example, understanding how environmental nonstationarity affects coexistence via the storage effect has implications for questions of how and when the storage effect may evolve and how species survive glaciation and other major climatic events. 


Nonstationarity in {\bf drivers}
\begin{enumerate}
\item Examples? $CO_{2}$ -- which is trend in mean, some example of shift in variance or such ...
\item Anything else here?
\end{enumerate}

Stationarity and nonstationarity in {\bf responses}, see {\bf Figure 4}
\begin{enumerate}
\item linear vs. nonlinear response functions (example with phenology, Primack vs. vernalization etc.) $\rightarrow$ tie to physiological level (err, now we're in scaling world)
\item species have different responses (nonstationarity across species? ack.)
\item species-specific responses should link up to how nonstationary drivers affect drivers of coexistence .... 
\end{enumerate}



%% 
%% Not sure if this fits somewhere (the evolving view of Chessonian coexistence? Or really may fit below)
This shift in perspectives on succession is one example of ecology's maturing view of the complexity of community assembly (Figure 2 `previous'). Early views of succession effectively assumed a simple environmental filter where climate was implicitly assumed to be stationary background noise, never providing any means to shift or alter trajectories; layered on this was the Hutchinsonian niche model where species coexisted via separate hyperdimensional moulds of resource use (which were often implicitly assumed to be effectively constant in space and time). A more modern view of succession includes an environmental filter that considers climate means as well as regular oscillations and events that may reset succession (Figure 2 `recent'). While researchers still sometimes use climatic niche distribution models to estimate environmental filters these is growing return to basic physiological limits that truly define the environmental filter for most species. Such physiological limits are most often defined by climate events and it is here that climate projections are moving. \footnote{Lizze: Add in Ayres beetle example.} Layered on this is increased recognition of the important variability in climate to coexistence: the Hutchinsonian niche is now one considered mode of coexistence, with increasing research in Chessonian coexistence mechanisms such as the storage effect highlighting the importance of temporal variability in shaping the biotic filter of community assembly. Chessonian coexistence has, in some ways, blurred the lines between the environmental and biotic filters of community assembly. \\

Understanding how both temporal and spatial dynamics may control communities is an important step toward climate change projections, but will critically require an additional layer of understanding how nonstationarity will shift these dynamics. The Dust Bowl was a major event for tallgrass plant communities---and typifies an important difference between many previously-studied climate events and the non-stationarity of current anthropogenic climate change. Following the Dust Bowl as climate returned to a state similar to that before the Dust Bowl, many plant communities also eventually returned to species assemblages similar to those before the Dust Bowl \citep{Weaver1936}. This is a rough prediction of the storage effect model, which assumes underlying stationarity in the environment and species responses to it. Climate change, however, is specifically a nonstationary event and predictions of what happens to communities under nonstationary environmental dynamics are not well-developed (see next section on `Contingency'). 
%%

% OLD 

While succession highlights that most ecological systems are inherently nonstationarity across long (century to millenia) timescales, many anthropogenic forces affecting ecological systems---for example, fire regimes that shift due to human ignition sources, damn-building phases in step with regional and country-level development---are nonstationary over much shorter timescales. Recent climate change, at least partially associated with greenhouse gases \citep{ipcc2013}, represents an excellent example of nonstationarity (Figure 6). For many systems, climate change has resulted in an obvious trend of increasing mean temperature, with variance about this mean often shifted equally \citep{Huntingford2013,Rhines2013}, though the concept of nonstationarity can result in shifts beyond the mean---including shifts in cycles or noise. \\

% Chronosequences
Several major approaches in ecology address temporal scaling issues. The first and perhaps most-widely applied method substitutes space for time (chronosequence, CITE Jenny 1941). Research in long-term ecosystem development relies on this approach [Jenny 1941], examining how the state factors of climate, organisms, relief, and parent material vary across space to understand long-term shifts in ecosystem properties, such as net primary productivity \citep{Wardle:2004wb}. This type of approach makes several major assumptions---including stationarity in the trajectory of ecosystems across space (see Figure X or some section below)---and requires extensive effort to locate relevant sites and pinpoint their temporal dimension, but provides tremendous power to ask questions on timescales far longer than the full lifespan of the field of ecology and, perhaps most powerfully, to project forward. 
% end % 

% JD's bunny story:
You jest, but this is perfect! Myxomatosis. We have a mix of slow and fast processes, rabbit generation times, and population cycles versus viral replication rates (http://www.jstor.org/stable/1943014).

Virus originally introduced into Australia  around 1950, initially had 99\% mortality rate, and at some sites rabbit pop dropped by 90\%. However, within a year mortality rate had dropped to 90\% and continued to decline and was <30\% within 8 yrs.

The increased resistance of the rabbit was due to the disease-induced mortality of genetically susceptible rabbits (i.e. selection on rabbit population), while the decline in the frequency of the more virulent strains of the virus depended on the interaction between the virulence of the virus and its transmissibility by the vector (mosquitos) - strains which were of intermediate virulence were transmitted more (too virulent and you kill the rabbit, not virulent enough and you produce too few viral particles). Emergent property of intermediate virulence as a function of viral evolution and selection on rabbit populations. In addition, results in periodic cycles in rabbit populations (from simulations) reflecting seasonal reproductive rate and host-pathogen dynamics (fig 6 in paper).

% JD's GWR stuff:

Spatial processes also demonstrate nonstationarity. Ignoring nonstationarity can misinform predictive models, which typically assume the parameters of a process (e.g. variance and mean) are independent of location and direction (Fortin and Dale, 2005). However, relaxing this assumption, for example, using geographical weighted regression (Brunsdon et al. 1996: http://onlinelibrary.wiley.com/doi/10.1111/j.1538-4632.1996.tb00936.x/pdf), which allows different relationships to exist at different points in space, can provide novel insights, helping discern process from pattern. GWR has proven particularly informative in the field of spatial macroecology as we have better mapped the distribution of animal and plant species richness across the surface of the Earth, and identified the major environmental axes with which it covaries. The latitudinal gradient in species richness is perhaps the most well-known biodiversity pattern, and the search for the underlying driver has been referred to as the holy grail of ecology (Huston M. A. 1994 Biological diversity: the coexistence of species on changing landscapes. Cambridge, UK: Cambridge University Press.). Competing explanations suggest the tropics as either a museum of diversity, where species richness has accumulated gradually over long times, or a cradle of diversity, in which high tropical species richness is explained by rapid diversification (Stebbins 1974: Stebbins G. L. 1974 Flowering Plants: Evolution Above the Species Level Harvard Univ. Press, Cambridge, MA). Fossil evidence suggests that the tropics might act as both a cradle and a museum (Jablonski et al. 2006: http://www.sciencemag.org/content/314/5796/102.short). By exploring non-stationarity in environmental predictors it is possible to reveal different richness predictors in the New World versus the Old World, suggesting explanations for the latitudinal gradient in species richness might differ between these two major biogeographical regions (see Davies et al. 2011; doi: 10.1098/rstb.2011.0018).
