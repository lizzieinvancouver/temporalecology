\documentclass[11pt,a4paper,oneside]{article}
\renewcommand{\baselinestretch}{1.4}
% \renewcommand*{\thefootnote}{\fnsymbol{footnote}}
\usepackage{sectsty,setspace} 
\usepackage[top=1.00in, bottom=1.0in, left=1in, right=1in]{geometry} 
\usepackage{graphicx}
\usepackage{epstopdf}
\usepackage{amsmath,latexsym,amssymb,wasysym}
\usepackage{natbib}
\usepackage{lineno}
\usepackage{todonotes}

\begin{document}
\bibliographystyle{/Users/Lizzie/Documents/EndnoteRelated/Bibtex/styles/amnat}

\noindent Title: \emph{Temporal ecology and the future of nonstationary systems} or \emph{The importance of temporal ecology to forecasting nonstationary systems} or (I won't let it quite die yet) \emph{Temporal ecology in the Anthropocene}\\
\\
% Building the discipline of temporal ecology
% Temporal ecology and the future of integrative climate change research
% Temporal ecology in a changing world (from Kendra, but I do not like this, I'd prefer 'nonstationary world' if we're doing world or 'shifting' instead of changing
% Temporal ecology in the anthropocene
% Temporal ecology in shifting landscapes/environments
\noindent E. M. Wolkovich,$^{1,2,3}$* B. I. Cook$^{4,5}$, K. K. McLauchlan$^{6,7}$ \& T. J. Davies$^{8}$\\
\\
\noindent \emph{$^{1}$Biodiversity Research Centre, University of British Columbia, Vancouver, BC, Canada; $^{2}$Arnold Arboretum, Boston, Massachusetts, United States of America; $^{3}$OEB, Cambridge, Massachusetts, United States of America; $^{4}$NASA Goddard Institute for Space Studies, New York, New York, United States of America; $^{5}$Ocean and Climate  Physics, Lamont-Doherty Earth Observatory, Palisades, New York, United States of America; $^{6}$ Department of Geography, Kansas State University, Manhattan, Kansas, United States of America; $^{7}$ University of Oxford, Merton College, Oxford, United Kingdom $^{8}$Department of Biology, McGill University, Montreal, Quebec, Canada}\\ %OX1 4JD for Oxford

\noindent *corresponding author: wolkovich@fas.harvard.edu, phone: 1.603.667.5099\\

\noindent Word count (total/allowed): XXXX/7500 in main text plus XXXX in boxes currently, XXX/XXX in abstract % Ideas and Perspectives: maximum of 7500 words (main text), 10 figures, tables or boxes, and 80 references.

%%%%%%%%%%%%%%%%%%%
%% Do the SUPPLEMENT! %%
%%%%%%%%%%%%%%%%%%%

% \newpage
\begin{abstract}
Two fundamental axes---space and time---shape ecological systems and their dynamics. Over the last 30 years a growing focus on the importance of space has developed alongside increasing rates of habitat fragmentation and loss. This extended focus on understanding the consequences of manipulated space for populations, species and ecosystems has resulted in an integrative, multidisciplinary science adept at building concepts and theory to address both basic and applied ecological questions. We argue here that increasing rates of current climate change---the effective manipulation of time by humans---is resulting in a similar opportunity and need to bridge disciplines and perspectives to build a modern, predictive framework for temporal ecology. Similarities between space and time provide the outline of a framework for forecasting ecological responses to altered temporal dynamics in ecosystems, while the unique attributes of time highight where new theory and approaches are needed. We emphasive two major areas that would benefit from increased study in order to build a more robust, predictive framework for temporal ecology---temporal scaling (from short to long timescales, and the reverse) and a greater appreciation nonstationarity in drivers and responses of ecological systems. We highlight how a renewed, intedisciplinary focus on time in ecology could coalesce related concepts, while guiding the development of new theories and methods to guide further data collection. This renewed focus could help answer fundamental questions of how temporal processes may structure ecological systems, from populations to ecosystems. From here, the next challenge for ecology will be to unite predictive frameworks from spatial and temporal ecology to build robust forecasts of when climate change and habitat degradation will pose the largest threats to species and ecosystems, as well as where and when the best opportunities exist for mitigation and conservation. 
\end{abstract}
% Ecologists have long embraced the diversity of temporal issues rampant in much environmental data, and often leverage space to elucidate how temporal processes may drive communities and ecosystems. Complexities in ecological time series and feedbacks inherent in biological systems mean that progress towards understanding how and to what extent systems are driven by temporal patterns require perspectives linking short and long-term dynamics, as well as relevant data. [Slow progress] is understandable given the unique attributes of time, methodological constraints and, until recently, dearth of available time series data. However, climate change, and other aspects of how humans have shaped ecological systems, have highlighted how omnipotent temporal dynamics are. We argue that ecological research would accelerate with a renewed, more constant temporal focus---across and between subfields and related disciplines. 

\tableofcontents

\newpage
\linenumbers

\noindent {\bf Introduction}\\

Thirty years ago a transformation in ecological thinking was underway, precipitated by questions of how anthropogenic habitat loss and fragmentation impacted populations, communities, and ecosystems. Addressing these questions required ecologists to work at 
scales far larger than their traditional plot sizes, statistical methods and theories allowed; it required integrating perspectives and methods from other disciplines (e.g., geography and evolution) while building upon and developing a body of theories (e.g., island biogeography, metapopulation) and concepts (edge effects and corridors). The rise of spatial ecology has lead to important advances in both applied and basic research. Over time the field has tackled major ecological questions by building from single-species metapopulation to multi-species metacommunity models \citep{Pillai2011} and from local to biogeographical scales \citep{bell2001}. As the field has matured, a suite of dedicated journals  (e.g. \emph{Diversity and Distributions, Journal of Biogeography}) have provided a forum for the exchange of ideas and cross-pollination between otherwise disparate ideas and disciplines. %\footnote{If possible and quick would be nice to add something about how this came alongside lots more spatial data, and the computing to handle it, though not quite sure what drove what there. But, whatever, we're missin the data angle here. Though I bring it in below, which I think may be enough.}\\

In the midst of this transformation, anthropogenic forces have also fundamentally shifted the temporal dynamics of most systems. From arctic to temperate biomes climate change has produced extended growing seasons---fundamentally altering how organisms in these habitats experience time, and challenging ecologists to make predictions of how such temporal shifts directly and indirectly alter populations and communities. Much as questions related to habitat fragmentation pressed ecologists to work at larger spatial scales, climate change and related issues have challenged ecologists to better understand temporal dynamics, over longer time scales, facilitated by improved integration of perspectives from evolution and paleobiology. Together, these shifts have rapidly yielded data at larger scales than previously-available, with climate change in particular precipitating the identification and study of a wealth of long-term datasets in addition to those already available. For example, the study of phenology---a discipline with over a 1000-year record of study---has received With increasing availability of long-term data, creeping timescale issues have arisen: population dynamics that appear more complex when examined in longer time series \citep{Ziebarth2010}, selection that weakens when integrated over longer studies \citep{Hendry1999,schoener2011}. Within disciplines vocabularies have developed to describe the importance of temporal dynamics, often producing differing terms for similar concepts (e.g., lag effects \emph{sensu} some Sala study and carry-over effects, \emph{sensu} Betini et al. 2013). Together, we argue, these factors have resulted in a current need for researchers to develop a renewed framework for temporal ecology---one that builds on growing needs and resources towards predictions for how shifting environments shape species, communities and ecosystems. Here we offer a starting point by reviewing the important attributes shared between temporal and spatial ecology, alongside the unique aspects of time that will require new perspectives and methods for robust forecasting. \\

\section{Fundamental and linked axes: Space and time}
Two major axes shape ecological systems---space and time. In turn these two axes have shaped many of the fundamental questions in ecology: How do spatial and temporal variation in the environment drive coexistence? How do past disturbances and succession combine to structure current landscape patterns? [We need more here.] Such questions highlight how entertwined temporal and spatial ecology are---and the two axes share many important similarities.\\

Similar to space, time---in ecology and across disciplines---is populated by conspicuous patterns. Many time series can be decomposed into regular cycles (e.g., daily, seasonal, multi-annual), longer-term trends and remaining `noise.'  Understanding to what degree regular fluctuations or cycles in ecological systems are shaped by external temporal patterns or are driven by ergodic properties of populations and species interactions makes up a large portion of study in behaviour \citep{macarthur1958}, physiology \citep{Lambers:2008jb}, population  \citep{May1976,Gurney1985,Krebs2001,Yang:2004zd} and community \citep{Chesson:1997dz} ecology.  Relatedly, the random variations outside of cycles and trends---temporal `noise'---have also received extensive study by both population \citep{Ripa1996,Kaitala1997,Bjornstad:1999kl} and community ecologists \citep{Chesson:2000vd}. Indeed, classical community ecology has used both intra- \citep{parrish1979,Albrecht:2001id} and interannual \citep{Chesson:1997dz} timescales to explain coexistence via temporal partitioning or small-scale differences in species' responses to a temporally-variable environment \citep{macarthur1958,Hutchinson:1961ui}. An additional similarity between space and time is the importance of scale in defining and understanding patterns and associated processes. Just as patterns may change when examined at local versus regional scales \citep[e.g.,][]{Fridley:2007ct}, trends may appear as cycles, and parts of cycles as singular events or noise, depending on scale. \\

While sharing some similarities with space, time is unique in several important aspects. First and despite great novels and films to the contrary, humans cannot directly manipulate time. While researchers have manipulated space at small \citep[e.g.,][]{huffaker} and large \citep[e.g.,][]{Terborgh:2001bw} scales with varying complexities, time can only be manipulated indirectly. Ecologists may adjust the timing of species' interactions \citep{Yang:2010cq} or underlying drivers of temporal processes, but they cannot fundamentally alter time to test its role in structuring systems. Next, time is directional. While space may have directional patterns (e.g., altitudinal and latitudinal trends) it is possible to view spatial patterns from almost endless directions and return to a place multiple times. In contrast, temporal patterns have a singular direction. Once an event has unfolded all following patterns and processes may be impacted by it without any temporal recourse to return to it or examine it in another direction. Finally, humans experience only a very tiny snapshot of time. Thus, while ecologists may cover the entire globe to map spatial patterns, temporal patterns governed by very short or very long timescales are inherently difficult for humans to study. For mortal ecologists temporal dynamics playing out over timescales longer than a career have never been easy. In the words of \citet{wiens1986},``[w]e get only a brief and often dim glimpse of the relevant processes.'' Together these unique aspects have shaped how ecologists have tended to view and approach time. \\

Clearly, ecologists have long embraced the diversity of temporal issues rampant in much environmental data, and have often exploited similarities between space and time to elucidate the defining history of many communities [CITE]. Yet climate change, and other aspects of how humans have shaped ecological systems, challenges ecology to move forward in time towards forecasting. We believe leveraging similarties between spatial and temporal dyanmics, while also recognizing unique attributes of time, will be critical to building a predictive framework. Two major areas that we see as those most critical to rapid advances in predictions---scaling and nonstationarity---include this mix of similarities and differences. We babble on about these below.

\section{Current hurdles to improved forecasting: Temporal scaling and nonstationarity}
\noindent \emph{We start with a short opener, that goes with the current first figure -- that is, {\bf Figure 1}, `Increasing availability of long-term records ..'.}\\

\begin{enumerate}
\item we need more work now in two key areas: scaling and nonstationarity
\item Temporal scaling: figuring out what scales matter to question and how they build on each other ... (to improve predictions)
\item Nonstationarity (what it is and why it matters to projections): note that stationarity is about stochastic processes
\end{enumerate}

\noindent \emph{Text parking lot:}\\
Climate change has, however, refocused ecological thinking on temporal scaling. It has provided a major new impetus to revisit fundamental questions whose answers have applications far beyond climate change: for example, understanding how environmental nonstationarity affects coexistence via the storage effect has implications for questions of how and when the storage effect may evolve and how species survive glaciation and other major climatic events. 

\begin{figure}[h!]
\centering
\noindent \includegraphics[width=0.5\textwidth]{/Users/Lizzie/Documents/git/manuscripts/temporalecology/figures/prediction/prediction.png}
\caption{Increasing availability of long-term records in ecology means many labs now possess repeated measures observational data of 5-10 years or more, and challenges researchers with how best to interpret trends in such data. Depending on the system and period of observation, what looks like a linear increase (a) could be part of a regular long-term cycles (b), indicative of a major shift in the system into a nonstationary period (c) or possibly part of both (d), especially if forcing on the system has changed---as seen in many systems with climate change.}
\end{figure}

\newpage
\begin{figure}[h!]
\centering
\noindent \includegraphics[width=1\textwidth]{/Users/Lizzie/Documents/git/manuscripts/temporalecology/figures/ecoevoclimate/climateEcoEvo_alltogethernowPlus.png}
\caption{Robust forecasting in temporal ecology requires recognizing the multiplicative dimensions of time inherent in most ecological processes (top arrows). For example, predictions of species' responses to climate change must consider both: (1) that many species experience far larger shifts in temperature on the timescale of hours to days (left) and (2) that over their evolutionary history many species have experienced climate swings similar in magnitude and rate (middle), in addition to the pressues of glaciation cycles (right). Differing methods in ecology (bottom) are optimized to differing timescales but ecologists are inherently most apt at working in timescales of days to years.}
\end{figure}

\subsection{Temporal scaling}
Scaling issues in temporal ecology in many ways mirror similar issues in spatial ecology, with many of ecology's most vociferous debates have routinely hinged on spatial or temporal scaling \citep{wiens1986}. Just as patterns observed at $1m^2$ plots often do not scale to landscapes, temporal dynamics at short timescales often do not appear to impact long-term dynamics [need example]. Many foundational questions in the field are still centered on the difficulties of dealing with time, including the struggles of incorporating fast and slow processes in models of ecosystem dynamics \footnote{Lizze adds ref} and temporal community coexistence via within year dynamics (e.g., phenology) versus between year dynamics \citep{Chesson:1997dz}.\\

Temporal scaling becomes particularly important as ecology moves towards temporal projections. For example, robust projections of how a species will respond to climate change must consider how a species' response to a persistent increase in mean temperatures building over many years will differ from much larger short-term fluctuations that the species may deal with on a daily or weekly basis {\bf Figure 2}---and whether responses across such timescales are linked. Relatedly, given that the average species age is [Jonathan inserts number and reference] the best projections would also consider how the species has responded to previous major climatic shifts, often in equal magnitude and rate with climate change {\bf Figure 2}. Such examples highlight the two major dangers in temporal scaling: working at too long a scale, versus working at too short a scale.\\

\noindent Topic: It's bad to work at too long a timescale, especially for prediction---because mechanisms often happen at the shorter timescales.
\begin{enumerate}
\item complain in one sentence about climate envelopes
\item Jonathan's complaint about niche evolution/climate change stuff could go here (?)
\item then show the better way: extremes versus means (Ayres beetles -- aggregating up to longer temporal scales wipes out the daily fluctuations that can kill things)
\end{enumerate}

\noindent Topic: It's bad to work at too short a timescale, especially for prediction---because longer timescales can often obliterate what matters at shorter scales.
\begin{enumerate}
\item working on a short timescale in an ENSO controlled system (Colorado River Compact) leads to crap predictions
\item Is there anything else to say here? A paleo example would be nice....
\end{enumerate}

\noindent Topic (or transition plus topic): Well, this is inconvenient---it's the Goldilocks of timescales, but it really depends on the question and system and you just have to be explicit, thoughftul and figure it out. An example of this is events, where the timescale really matters and depends on the question/system.
\begin{enumerate}
\item Event messiness: quick review of conventional use of this term in ecology (what is that?) and then transfer to box ... Events are the messiest of concepts in time (see Box 1: The trouble with events) encompassing slow processes that trigger rapid transitions \citep{Foley2003} and relatively rapid occurrences that produce long-term shifts \footnote{Ben, recommend citation?}, as well as extreme events, which are generally defined as falling beyond a statistically-defined threshold \citep{ipccextreme2012}.
\item mention of frosts versus Younger Dryas (I have no idea where we were going here, but I think we need more here)
\item Three (3) features of detecting events (see {\bf Figure 3}):
\begin{enumerate}
\item sampling (temporal) frequency
\item duration of sample/record
\item peak height (where on tail are you? Black swan events)
\end{enumerate}
\end{enumerate}
 

\noindent \emph{Some extra text I deemed useful enough to park here:}\\
Across many areas of ecology---with some notable exceptions---there is often a hidden rift between the relevant timescale for a given ecological question and the timescale of methods applied to address it.\\

For, while anthropogenic climate change is a dramatic alteration to the earth's climate system, equally rapid shifts in climate on similar timescales have occurred before and equal shifts occur commonly in the environments of most organisms on far shorter timescales.\\

Today, ecology has an additional step-up in reapproaching questions of temporal scaling from greater integration with other disciplines. Just as temporal scales inherently define and link subfields in ecology (Figure 3) they also bridge ecology to climate science and evolution. 

Several major approaches in ecology address temporal scaling issues. The first and perhaps most-widely applied method substitutes space for time (chronosequence, CITE Jenny 1941). Research in long-term ecosystem development relies on this approach [Jenny 1941], examining how the state factors of climate, organisms, relief, and parent material vary across space to understand long-term shifts in ecosystem properties, such as net primary productivity \citep{Wardle:2004wb}. This type of approach makes several major assumptions---including stationarity in the trajectory of ecosystems across space (see `Contingency' section below)---and requires extensive effort to locate relevant sites and pinpoint their temporal dimension, but provides tremendous power to ask questions on timescales far longer than the full lifespan of the field of ecology and, perhaps most powerfully, to project forward. Another very basic approach is the comparison of methods, sometimes abstracted into experiments, observations, long-term observations and modeling \citep{Carpenter:1992hk}. Experiments are often conducted on the shortest timescales---from days to weeks in the lab, to weeks and years in the field---and, as discussed above, may only capture transient dynamics. They allow, however, generally the most powerful tests of mechanisms and, relatedly, major insights \footnote{Anyone have a suggested reference here? I'd like one that makes experiments look good}. Such powerful tests are buoyed by comparisons with observational data, even short-term. For example, a experimental study of the impact of an invasive species by manipulating its density may find certain effects: showing similar effects along a gradient of low to high abundance of the invasive species provide support that results are not due to experimental artifacts. Showing congruence with long-term data is a further boost and using modeling to extend the scale of the findings---possibly incorporating important longer-term dynamics such as climate cycles---can greatly improve the utility of the original experiment as well as most probably providing additional insights and new predictions to test. Related to this, a greater incorporation of a longer-term temporal dimensions to short-term studies of long-term processes may come from incorporating perspectives of disciplines traditionally focused on long timescales. For example, paleo ecology has provided major insight and complements to modern studies of current climate change \footnote{Kendra, please fix}. While something about eco-evo could go here \footnote{JD, please fix}. \\

One additional simple method to clarify the temporal dimensions needed to address particular ecological questions is to fully recognize the timescales at play. This means giving all the relevant known timescales---these include the generation times of the study organisms, frequency of disturbance, as well as the period of climate oscillations---then placing the study in the relevant part of these cycles. ...\\

Thoughtful discussion of temporal issues---or at a minimum giving relevant temporal dimensions---can enhance a project with cascading benefits for the field of study. Indeed, major debates in ecology have often arisen from hidden timescale issues. For example, in studies of trophic cascades, freshwater systems with rapid turnover times often yielded results in accordance with equilibrium expectations, while studies in terrestrial systems---where experiments were much shorter than many of the manipulated species' generation times---often yielded contrasting results due to transient dynamics \citep{Cebrian:2009hg}.\\ 

The above-described approaches are not new in ecology. It is not that ecologists do not use these methods, it is that they have been unevenly applied across time and across subfields such that the inherent complexities time introduces to ecology have not always been transparent. A more even application of these approaches would benefit ecology, but will require better training of students in the above approaches and higher expectations of approaching all ecological questions with a more overtly temporal perspective. This will mean continually cross-checking approaches of varying timescales, modeling studies to extend beyond currently available data, and more integration of disciplines that have sometimes worked separately because of their underlying disparate timescales.... Such a shift is especially needed given the current rates of nonstationary processes, such as climate change, operating in many systems. \\


\begin{figure}[h!]
\centering
\noindent \includegraphics[width=0.8\textwidth]{/Users/Lizzie/Documents/git/manuscripts/temporalecology/figures/extremes/extremes.png}
\caption{Both the temporal frequency (a) and duration (b) of sampling affect the utility of time series data to understand how climate impacts ecological dynamics. In plant phenology research, within-year sampling is sometimes focused mainly on the start of spring, with fewer observations of later season events. Such focused sampling may give very different answera than longer within-year sampling, which encompasses additional species. For example, 2012 was an extremely warm spring for much of northeastern North America and early season species, such as \emph{Prunus serotina}, flowered much earlier than ever previously observed (a, z-scored data show 2012 was three standard deviations beyond the mean for this species, inset shows raw time series data), while many other species, such as \emph{Betula lenta} that flower several weeks later, did not show as extreme a response. Duration of sampling can also easily affect how extreme an event---such as a droght---appears. Considering only the 20th century (b, see light blue bars in histogram, and light blue shaded area of inset), a 4 year drought in central North America appears very unique from the usual 1-2 year length of most droughts. Extending the record, however, to the last 1,000 years (gray bars in histogram) shows far longer droughts throughout the 12th to 15th centuries. [Need to give references for these data!]}
\end{figure}

\newpage

\subsection{Nonstationarity in ecological systems}
\begin{enumerate}
\item Nonstationarity occurs in space: GWR
\item Define (already in paper): also change that affects \emph{ability to predict/project} ... Trends represent one type of nonstationarity. Stationarity---the concept of any process with a probability distribution that does not change when shifted in time---is a major assumption of many time series methods and---we argue---ecological concepts and theory. Stationarity allows for extensive variation about a mean including possibly cycles and noise, but requires an unchanging underlying distribution such that there are no trends in the mean, cycles or noise. Nonstationarity also allows for the possibility of cycles and noise but, in contrast, it requires shifts---in the mean, cycles and/or noise---across time. Recent climate change, at least partially associated with greenhouse gases \footnote{CITE new IPCC, does anyone have the correct citation information for this?}, represents an excellent example of nonstationarity (Figure 1): for many systems, climate change has resulted in an obvious trend of increasing mean temperature, with variance about this mean shifted equally \footnote{Ben, recommend citation?}. 

Over recent decades ecology has shifted increasing focus to nonstationary patterns and events, which are often coupled. Concepts of nonstationarity have been long-studied in certain arenas of ecological thinking: succession---changes in the structure and function of ecosystems over time---is fundamentally a prediction of nonstationarity and one of the basic concepts of ecology \citep{clementsbook,gleason1926}. This nonstationarity is in turn invoked across studies of community \citep{Levin:1992rg} and paleoecology \footnote{Kendra, please suggest ref} that explain diverse landscape patterns via events---often disturbances---that reset successional clocks. How systems respond to events is also foundational across ecology in studies of resilience, and other definitions of stability \citep{Grimm:1997}. Concepts of systems with multiple stable states \citep{yang2010}, another version of nonstationarity, also generally cite events that may shift systems from one state to another. Such events may be either rapid \footnote{Ben, suggest ref?} or slow forcing events \citep{Foley2003}, such as recent anthropogenic climate change. Additionally, many other anthropogenic forces affecting ecological systems---for example, fire regimes that shift due to human ignition sources, damn-building phases in step with regional and country-level development---are nonstationary. In contrast, many underlying assumptions of ecological theory and basic approaches are not.\\


\item {\bf Classical ecological examples} of temporal nonstationarity: succession and regime shifts
\item Some major {\bf assumptions} of stationarity in ecological theory that are being revisited
\begin{itemize}
\item species distribution models
\item non-linear physiological responses (not sure if this still goes here)
\item shifting stressors (reapproaching multiple stressors as Leibig's low of the minimum $\rightarrow$ dominant stressors, directionality in $CO_{2}$ and N, or drought etc..
\item also, island biogeography, anything from evolution, or paleo?
\end{itemize}
\end{enumerate}

Nonstationarity in {\bf drivers}
\begin{enumerate}
\item Examples? $CO_{2}$ -- which is trend in mean, some example of shift in variance or such ...
\item Anything else here?
\end{enumerate}

Stationarity and nonstationarity in {\bf responses}, see {\bf Figure 4}
\begin{enumerate}
\item linear vs. nonlinear response functions (example with phenology, Primack vs. vernalization etc.) $\rightarrow$ tie to physiological level (err, now we're in scaling world)
\item species have different responses (nonstationarity across species? ack.)
\item species-specific responses should link up to how nonstationary drivers affect drivers of coexistence .... 
\end{enumerate}

\begin{figure}[h!]
\centering
\noindent \includegraphics[width=1\textwidth]{/Users/Lizzie/Documents/git/manuscripts/temporalecology/figures/nonstationarity/nonstationarity.png}
\caption{Forecasting ecological responses to climate change requires layering projections of complex physiological responses onto nonstationary drivers, such as increasing temperatures (here, we show annual northern hemisphere temperature for ocean and land shown with a 10-year locally weighted scatterplot smoothing in red). Much research has focused on projecting phenological responses to this increasing temperature (b), which requires first understanding how such annual mean changes in temperature translate to impacting daily temperatures (see $y$ axis in a, versus $x$ axis in b). Next, in their simplest forms, responses could either be linear (b.1) where higher daily temperatures yeild earlier leafing or flowering, for example, or non-linear (b.2) if responses to temperature are limited at higher values by additional factors (here we consider photoperiod as an example, though drought, nutrients or other factors could also be critical). We stress that these two examples (b.1 and b.2) are still highly simplified versions of the underlying physiological process driving spring phenology in many species.}
\end{figure}

Grand challenge: Interaction of predicting nonstationarity in drivers $x$ responses (see {\bf Figure 5})
\begin{enumerate}
\item multiple stressors and predictions (wolves and climate change, cite Laube---multiple issues and scaling experiments etc.)
\item Big section on linking work in assembly theory (priority effects, Chase, Fukami etc.) and multiple trajectories, historical contingency
\item End on upbeat note (note to self, Lizzie, be upbeat)
\end{enumerate}

\begin{figure}[h!]
\centering
\noindent \includegraphics[width=0.4\textwidth]{/Users/Lizzie/Documents/git/manuscripts/temporalecology/figures/assembly/assembly.png}
\caption{The evolution of how ecologists view time can be seen partly in the maturation of successional theory. Early work (a) tended to focus on one possible trajectory and outcome. A consistent, predictable turnover of species was the main driver of ecosystem development; mean climatic factors shaped the species pool on the largescale, but climate was otherwise generally unimportant. As work progressed (b), ecologists recognized that multiple trajectories were possible---often triggered by disturbances that fundamentally reset the temporal position of an ecosystem along its development curve. Disturbances includd a variety of factors related to climatic extremes (e.g., drought, fire) and highlighted a growing recognition that climate's variability could impact succession. More recently (c), ecologists have layered onto this an appreciation of factors that may yield diverse trajectories and endpoints. Research on priority effects and other small discrepancies in initial conditions between otherwise similar sites have shown that such differences can result in different trajectories and endpoints. Additionally, research on tipping points and alternative stable states have highlighted that some events may transition ecosystems to fundamentally different states; nonstationarity in climate, or other ecosysten drivers, may contribute to such tipping points as it could be difficult for systems to return to trajectories if the underlying climate has shifted significantly while the system has been recovering from the disturbance.}
\end{figure}


\section{Thinking about space and time together is the next big frontier}
\begin{enumerate}
\item conservation: Potential kumbaya moment: conservation is both thinking about habitat loss (space) and climate change (time). Though focus has been mainly spatial. Good opportunity for habitat loss people to stop squawking about all the press climate change is getting.
\item think more nuanced that source-sink metapopulation stuff: need to add nonstationarity and temporal scaling to conservation of patchy landscapes 
\end{enumerate}

\noindent {\bf Placeholder text that Lizzie needs to go through:}\\
Recent advances in temporal ecology have come from the rise of spatial ecology and its emphasis in recognizing and understanding hidden dimensions in ecological models and theory. Decades of efforts in understanding the consequences of when space is made explicit have resulted in a return to the importance of temporal dimensions of many spatial models. For example, island biogeography theory predicts species richness based on several basic spatial metrics---but temporal dimensions of the controlling processes---immigration, extinction and speciation---also clearly govern predictions of species richness (CITE WIENS). Similarly, disease models have advanced through incorporating both spatial and temporal models of traveling waves \citep{Grenfell:2001ox} as disease prevalence varies both with population density and temporal fluctuations in that density \citep{Grenfell:2001ox} and climate \footnote{JD, please suggest ref}. \\

Such advances represent, however, only a small foray into the applications of fully embracing the interconnectedness of spatial and temporal dynamics in ecology. Coexistence theory---long stymied by models that required \(n\) different axes to produce \(n\) coexisting species alongside empirical examples of many co-occurring species that appeared quite similar when examined from one snapshot in the lab or field---advanced several football fields beyond their general stronghold at the 10-yard line when the role of variability in species responses to the temporal dimension was re-examined \citep{Chesson:1997dz}. Under the storage effect model \citep{Chesson:1997dz} species with identical nutrient uptake curves, identical predators and predation rates can co-exist via small differences in how they respond to temporal variability in their environment. Given the tremendous diversity of temporal variation in most environments it should not be surprising that it would be a major axis on which to mute competitive forces and thus promote coexistence. Since its introduction the storage effect model has been ported to spatial dimensions as well---where species coexist via reduced competition from spatial variability. Tests for such models have found support separately for temporal \citep{Angert:2009} and spatial \footnote{Lizzie adds ref} storage effects in studied communities, but we expect most communities function based on a constantly shifting mix of the two mechanisms (in addition to fluctuation-independent mechanisms). For example, studies of community change during the Dust Bowl show how dramatically dominant species may decline to apparent local extinction while rare species rise to abundance \citep{Weaver1936}. Recent work, however, highlights the persistence of rare species throughout long time periods through landscape dynamics where microclimates maintain great species diversity \cite{Craine2012}. In such cases temporal storage effects are built on buffered population growth maintained by spatial dynamics; this is in addition to the more traditionally-considered temporal buffering via seedbanks or other long-lived lifestages of species. \\

Understanding how both temporal and spatial dynamics may control communities is an important step to climate change projections, but will critically require an additional layer of understanding how nonstationarity will shift these dynamics. The Dust Bowl was a major event for tallgrass plant communities---and typifies an important difference between many previously-studied climate events and the non-stationarity of current anthropogenic climate change. Following the Dust Bowl as climate returned to a state similar to that before the Dust Bowl, many plant communities also eventually returned to species assemblages similar to those before the Dust Bowl \citep{Weaver1936}. This is a rough prediction of the storage effect model, which assumes underlying stationarity in the environment and species' responses to it. Climate change, however, is specifically a nonstationary event and predictions of what happens to communities under nonstationary environmental dynamics are not well-developed (see next section on `Contingency'). The consequences of nonstationarity in temporal dimensions in most ecological models are unknown but critical for projection. For example, island biogeography makes predictions assuming stationarity in the mainland population, yet most mainland populations in recent times are rarely stationary. Additionaly, robust projections of climate change impacts on populations and species will require a nonstationary adjustment to the most classic spatial metaphor for a temporal process: adaptive landscapes. Ignoring species interactions and considering only shifts to the abiotic environment, climate change has resulted in rapid and effectively continuous shifts to most populations adaptive peaks and valleys. Climate change has thus highlighted how rapid evolution may be and bringing it firmly into an ecological timescale, but theory to how such nonstationarity may affect evolutionary outcomes has not kept pace \footnote{Lizzie add ref}.\\



\noindent {\bf Acknowledgments}\\
We thank both Monsters and Men. Conversations with D. Schluter and M. Donohue made my life better and helped this manuscript. NSERC BRITE funding.


\clearpage
\begin{figure}[h!]
\centering
\noindent \includegraphics[width=0.6\textwidth]{/Users/Lizzie/Documents/git/manuscripts/temporalecology/figures/events/events.png}
\caption{Forcings in ecological systems may be discrete, transitory events such as fire or insect defoliation events, but the ecological consequences may be transient and short-lived [1], or persist long after the forcing has disappeared or reversed [2]. Conversely, the forcings my be continuous and persistent, such as global climate change or the establishment of an invasive species. Similary, however, these continuous forces may give rise to transient ecological responses [3] or continuing responses [4]. Whether the ecological response is transient or continuous depends not only on the nature of the forcing, but also on the inherent capacity for resistance, resilience, and feedbacks within the ecosystem or community of interest.}
\end{figure}

\noindent {\bf Defining events}\\
\noindent With the increasing availability of long-term environmental and ecological data from the historical period and paleorecord, ecology is better poised now than ever before to improve understanding of temporal dynamics in ecological systems. One area of temporal dynamics that is of particular interest is how quickly and persistently ecosystems respond to forcings that may be either short-lived, \emph{transient} events (e.g. fires, droughts, insect defoliation, etc) or persistent, \emph{continuous} changes in the background state (e.g., climate change, introduction of invasive species, habitat fragmentation, etc, see Figure). Importantly, the permanence and velocity of ecological responses depends not only on the nature of the forcing (e.g., its severity and duration), but also on the internal dynamics and feedbacks within the ecosystems or communities themselves.\\

\indent For example, vegetation may quickly return to its previous state following transient disturbances, such as a fast growing grassland recovering after a fire or drought (e.g., ref XXXX), or a plant down-regulating initial photosynthetic enhancement in response to elevated CO2 concentrations (ref XXXX). Both responses can be considered transient, regardless of the nature of the forcing, and may indicate either some inherent resilience in ecosystem structure and function (in the grassland example), or fundamental shifts in the in the importance of the resource limitation and environmental stressor space. Ecosystems may also respond in persistent ways to either transient or continuous forcings. A relatively recent example is the switch from a ponderosa pine forest to a piñon–juniper woodland in Southwest North America following a major drought in the 1950s \citep{Allen1998}. This new woodland persists to this day, despite a subsequent return to more normal moisture conditions. And during the Mid-Holocene, the Sahara permanently shifted from a woodland savanna to a hyper-arid desert in response to changes in Northern Hemisphere summer insolation, with the ecosystem collapse happened much more quickly than the forcing change \citep{Foley2003}. Clearly, the nature of forcing events (fast or slow, discrete or continuous) does not necessarily map clearly onto ecological responses, presenting a challenge for better prediction of the speed and persistence of ecosystem responses.\\

\indent Additional difficulties may be presented by a special class of events known as `Black Swans'. A Black Swan event is defined by two components: (1) that it has dramatic effects on the system, but is extremely rare, such that (2) it is effectively impossible to predict using current methods. These two components lead to the third aspect of Black Swan theory: owing to a combination of their unpredictability and their large impact on the system there is a strong tendency to believe such events can be predicted---when, instead, by their extreme rarity and unpredictability this is generally impossible. Already there is evidence for ecologically important `Black Swan' events, including an 18th century drought in Eastern North America that has shaped successional trajectories to this day (ref XXXX) and XXXXXX. Detecting these events, and their importance for ecological processes, remains difficult.\\ 

\indent To achieve the goal of better understanding the importance and time scale of forcings and ecological responses, ecological modeling may benefit increasingly from probabilistic, rather than deterministic, modeling approaches, such as those used in the field of climate modeling (ref XXXX). Using a probabilistic approach allows for better elucidation of the internal, unpredictable variability in the system and the full exploration of the possible model parameter space. For example, many climate model projections use an ensemble approach, where individual ensemble members (i.e., climate projections) start from different initial conditions but all use the same set of forcings and boundary conditions (e.g., land cover, greenhouse gas concentrations, etc). In this way a spread of projections is generated, with probabilities of future climate states emerging naturally from the ensemble as a function of the forced response (ensemble average) and internal variability (ensemble spread). Probabilistic sampling and modeling may also allow for detection and attribution of controversial topics in ecology for which data are limited, such as CO2 fertilization (ref XXX) and invasive species (ref XXXX), as well as understanding the importance of very rare, but random events, like Black Swans.


\begin{footnotesize}
{\def\section*#1{}
\bibliography{/Users/Lizzie/Documents/EndnoteRelated/Bibtex/LizzieMainMinimal}
}
\end{footnotesize}

\end{document}



