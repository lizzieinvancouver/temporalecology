\documentclass[11pt,a4paper,oneside]{article}
\renewcommand{\baselinestretch}{1.4}
% \renewcommand*{\thefootnote}{\fnsymbol{footnote}}
\usepackage{sectsty,setspace} 
\usepackage[top=1.00in, bottom=1.0in, left=1in, right=1in]{geometry} 
\usepackage{graphicx}
\usepackage{epstopdf}
\usepackage{amsmath,latexsym,amssymb,wasysym}
\usepackage{natbib}
\usepackage{lineno}
\usepackage{todonotes}

\begin{document}
\bibliographystyle{/Users/Lizzie/Documents/EndnoteRelated/Bibtex/styles/amnat}

\noindent Title: \emph{Temporal ecology in the Anthropocene}\\
\\
% Temporal ecology and the future of nonstationary systems
% The importance of temporal ecology to forecasting nonstationary systems
% Building the discipline of temporal ecology
% Temporal ecology and the future of integrative climate change research
% Temporal ecology in a changing world (from Kendra, but I do not like this, I'd prefer 'nonstationary world' if we're doing world or 'shifting' instead of changing
% Temporal ecology in the anthropocene
% Temporal ecology in shifting landscapes/environments
\noindent E. M. Wolkovich,$^{1,2,3}$* B. I. Cook$^{4,5}$, K. K. McLauchlan$^{6,7}$ \& T. J. Davies$^{8}$\\
\\
\noindent \emph{$^{1}$Biodiversity Research Centre, University of British Columbia, Vancouver, BC, Canada; $^{2}$Arnold Arboretum, Boston, Massachusetts, United States of America; $^{3}$OEB, Cambridge, Massachusetts, United States of America; $^{4}$NASA Goddard Institute for Space Studies, New York, New York, United States of America; $^{5}$Ocean and Climate  Physics, Lamont-Doherty Earth Observatory, Palisades, New York, United States of America; $^{6}$ Department of Geography, Kansas State University, Manhattan, Kansas, United States of America; $^{7}$ University of Oxford, Merton College, Oxford, United Kingdom $^{8}$Department of Biology, McGill University, Montreal, Quebec, Canada}\\ %OX1 4JD for Oxford

\noindent *corresponding author: wolkovich@fas.harvard.edu, phone: 1.617.384.5494\\

\noindent Word count (total/allowed): 5751/7500 in main text plus 607 in boxes currently, {\bf 270/200 in abstract} % Ideas and Perspectives: maximum of 7500 words (main text), 10 figures, tables or boxes, and 80 references.

%%%%%%%%%%%%%%%%%%%
%% Do the Supporting Online Information! %%
%% see http://onlinelibrary.wiley.com/journal/10.1111/(ISSN)1461-0248/homepage/ForAuthors.html %%
%% Number 20 %%
%%%%%%%%%%%%%%%%%%%

% \newpage
\begin{abstract}
Two fundamental axes---space and time---shape ecological systems and their dynamics. Over the last 30 years a growing focus on the importance of space has developed alongside increasing rates of habitat fragmentation and loss. This extended focus on understanding the consequences of manipulated space for populations, species and ecosystems has resulted in an integrative, multidisciplinary science adept at building concepts and theory to address both basic and applied ecological questions. We argue that current
 increasing rates of climate change---the effective manipulation of time by humans---is resulting in a similar opportunity---and need---to build a modern, predictive framework for temporal ecology. Similarities between space and time provide an outline for forecasting ecological responses to altered temporal dynamics in ecosystems, while the unique attributes of time highlight where new theory and approaches are needed. We emphasize two major areas that would benefit from increased study to build a more robust, predictive framework for temporal ecology---temporal scaling and improved understanding of nonstationarity in drivers and responses of ecological systems. We highlight that a renewed, interdisciplinary focus on time in ecology could coalesce related concepts, direct the development of new theories and methods, and thus guide further data collection. The next challenge for ecology will be to unite predictive frameworks from spatial and temporal ecology to build robust forecasts of when climate change and habitat degradation will pose the largest threats to species and ecosystems, as well as identifying the best opportunities exist for mitigation and conservation. 
\end{abstract}
% Ecologists have long embraced the diversity of temporal issues rampant in much environmental data, and often leverage space to elucidate how temporal processes may drive communities and ecosystems. Complexities in ecological time series and feedbacks inherent in biological systems mean that progress towards understanding how and to what extent systems are driven by temporal patterns require perspectives linking short and long-term dynamics, as well as relevant data. [Slow progress] is understandable given the unique attributes of time, methodological constraints and, until recently, dearth of available time series data. However, climate change, and other aspects of how humans have shaped ecological systems, have highlighted how omnipotent temporal dynamics are. We argue that ecological research would accelerate with a renewed, more constant temporal focus---across and between subfields and related disciplines. 

\newpage
\linenumbers

% \noindent {\bf Introduction}\\
Thirty years ago a transformation in ecological thinking was underway, precipitated by questions of how anthropogenic habitat loss and fragmentation impacted populations, communities, and ecosystems. Addressing these questions required ecologists to work at scales far larger than their traditional plot sizes, statistical methods and theories allowed; it required integrating perspectives and methods from other disciplines (e.g., geography and evolution) while building upon and developing a body of theories (e.g., island biogeography, metapopulation) and concepts (edge effects and corridors). Over time yhe rise of spatial ecology has lead to important advances in both applied and basic research. The field has tackled major ecological questions by building from single-species metapopulation to multi-species metacommunity models \citep{Pillai2011} and from local to global scales \citep{bell2001}. As the field has matured, a suite of dedicated journals  (e.g. \emph{Diversity and Distributions, Journal of Biogeography, Landscape Ecology}) have provided a forum for the exchange of ideas and cross-pollination between otherwise disparate ideas and disciplines. %\footnote{If possible and quick would be nice to add something about how this came alongside lots more spatial data, and the computing to handle it, though not quite sure what drove what there. But, whatever, we're missing the data angle here. Though I bring it in below, which I think may be enough.}\\

In the midst of this transformation, anthropogenic forces have also fundamentally shifted the temporal dynamics of most systems. From arctic to temperate biomes climate change has produced extended growing seasons---fundamentally altering how organisms experience time, and challenging ecologists to make predictions of how such temporal shifts directly and indirectly alter populations and communities. Much as questions related to habitat fragmentation pressed ecologists to work at larger spatial scales, climate change and related issues have challenged ecologists to better understand temporal dynamics, over longer time scales, facilitated by improved integration of perspectives from evolution and paleobiology. Together, these shifts have rapidly yielded data at larger scales than previously-available, with climate change in particular precipitating the identification and study of a wealth of long-term datasets in addition to those already available. For example, researchers studying phenology have brought together hundreds of thousands of time-series datasets to understand the impact of climate change on the timing of animal and plant life cycle events \citep{Menzel:2006sq,Parmesan:2007tv}.\\

Yet, with the increasing availability of long-term data, creeping timescale issues have arisen: population dynamics that appear more complex when examined in longer time series \citep{Ziebarth2010}, selection that weakens when integrated over longer periods \citep{schoener2011,Uyeda2011}. Within disciplines vocabularies have developed to describe the importance of temporal dynamics, often producing differing terms for similar concepts (e.g., lag effects \emph{sensu} some Sala study and carry-over effects, \emph{sensu} Betini et al. 2013). Together, we argue, these factors have resulted in a current need for researchers to develop a renewed framework for temporal ecology---one that builds on growing needs and resources and works towards predictions for how shifting environments shape species, communities and ecosystems. Here we offer a starting point by reviewing the important attributes shared between temporal and spatial ecology, alongside the unique aspects of time that will require new perspectives and methods for robust ecological forecasting. \\

\noindent {\bf Fundamental and linked axes: Space and time}\\

Two major axes shape ecological systems---space and time. In turn these two axes have shaped many of the fundamental questions in ecology: How do spatial and temporal variation in the environment drive coexistence? How do past disturbances and succession combine to structure current landscape patterns? Such questions highlight how intertwined temporal and spatial ecology are---and the two axes share many important similarities.\\

Similar to space, time---in ecology and across disciplines---is populated by conspicuous patterns. Many time series can be decomposed into regular cycles (e.g., daily, seasonal, multi-annual), longer-term trends and remaining `noise.'  Understanding to what degree regular fluctuations or cycles in ecological systems are shaped by external temporal patterns or are driven by ergodic properties of populations and species interactions makes up a large portion of study in behavior \citep{macarthur1958}, physiology \citep{Lambers:2008jb}, population  \citep{May1976,Gurney1985,Krebs2001,Yang:2004zd} and community \citep{Chesson:1997dz} ecology.  Relatedly, the random variations outside of cycles and trends---temporal `noise'---have also received extensive study by both population \citep{Ripa1996,Kaitala1997,Bjornstad:1999kl} and community ecologists \citep{Chesson:2000vd}. Indeed, classical community ecology has used both intra- \citep{parrish1979,Albrecht:2001id} and interannual \citep{Chesson:1997dz} timescales to explain coexistence via temporal partitioning or small-scale differences in species' responses to a temporally-variable environment \citep{macarthur1958,Hutchinson:1961ui}. An additional similarity between space and time is the importance of scale in defining and understanding patterns and associated processes. Just as patterns may change when examined at local versus regional scales \citep[e.g.,][]{Fridley:2007ct}, trends may appear as cycles, and parts of cycles as singular events or noise, depending on scale (see Figure 1). \\

Time is unique from space, however, in several important aspects. First and despite great novels and films to the contrary, humans cannot directly manipulate time. While researchers have manipulated space at small \citep[e.g.,][]{huffaker} and large \citep[e.g.,][]{Terborgh:2001bw} scales with varying complexities, time can only be manipulated indirectly. Ecologists may adjust the timing of species' interactions \citep{Yang:2010cq} or underlying drivers of temporal processes, but they cannot fundamentally alter time to test its role in structuring systems. Next, time is directional. While space may have directional patterns (e.g., altitudinal and latitudinal trends) it is possible to view spatial patterns from almost endless directions and return to a place multiple times. In contrast, temporal patterns are arrow-like---they have each a singular direction. Once an event has unfolded all following patterns and processes may be impacted by it without any temporal recourse to return to it or examine it in another direction. Finally, humans experience only a very tiny snapshot of time. Thus, while ecologists may cover the entire globe to map spatial patterns, temporal patterns governed by very short or very long timescales are inherently difficult for humans to study. In the words of \citet{wiens1986},``[w]e get only a brief and often dim glimpse of the relevant processes.'' \\

Ecology has long embraced the diversity of temporal issues rampant in the discipline's defining questions, and has often exploited similarities between space and time to elucidate the defining history of many communities \citep{wiens1986}. Yet as climate change, and other aspects of how humans have shaped ecological systems challenges ecology to move towards forecasting we believe a stronger framework for temporal ecology is critical. Progress towards this goal will require leveraging similarities between spatial and temporal dynamics, while also recognizing unique attributes of time. It will of course additionally require identifying areas where advances could result in the greatest progress towards understanding and onward to prediction. Two major areas that we believe are most critical to rapid advances in predictions are temporal scaling and nonstationarity. \\

\noindent {\bf Current hurdles to improved forecasting: Temporal scaling and nonstationarity}\\

The increasing availability of long-term records in ecology provides a major opportunity to build an improved framework for temporal ecology and ecological forecasting. Specifically, these new data highlight two of the dominant hurdles that currently hinder accurate predictions: temporal scaling and nonstationarity (Figure 1). Temporal scaling refers to a broad suite of issues, including how processes with differing inherent timescales may interact, how species may respond to the same forcing over different time intervals (e.g., daily versus annual versus interannual temperature fluctuations; changes in extreme events versus the mean), and how much data are needed to draw conclusions regarding trends, variability, and periodicities. Further, in light of current, largely unprecedented, levels of anthropogenic disturbance, skillful forecasting will also require a consideration of how stationary biotic and abiotic processes and interactions will be in the future. Together, issues of scaling and stationarity raise the question of whether inferences drawn from historical data are appropriate for a future world with quantitatively different boundary conditions.\\

\noindent \emph{Temporal scaling}\\
Scaling issues in temporal ecology mirror many of the same challenges highlighted in spatial ecology. For example, while spatial ecology is concerned with how well observations at the plot level ($10^{1}m^2$s) scale to the landscape level ($10^{2}--10^{3}m^2$s), temporal ecology must harmonize across ecological processes that span disparate temporal scales, from minutes (e.g., photosynthesis) to days and weeks (e.g. phenology, annual productivity) and upward to centuries and millennia (e.g., successional dynamics, carbon cycle dynamics, evolution of species' niches). Indeed, issues with temporal dynamics observed at short timescales that do not appear to impact long-term dynamics have fueled many of ecology's most vociferous debates \citep{wiens1986}. This is illustrated, for example, by the difficulties inherent in incorporating fast and slow processes in models of ecosystem dynamics \citep{Carpenter2000} or modeling temporal community coexistence via both seasonal (e.g., phenology) and interannual environmental dynamics \citep{Chesson:1997dz}. Climate change has, however, refocused ecological thinking on temporal scaling, providing a major new impetus to revisit fundamental questions and to fully recognize where scaling issues limit robust predictions over longer time scales than have been typically been considered (years to decades to centuries).\\

Temporal scaling becomes particularly important as ecology moves towards temporal projections with global change. Such predictions generally require bridging from the shorter, finer temporal scales of fundamental ecological responses to both the longer evolutionary timescales that have shaped species' responses and to the often coarser temporal data available for model calibration and prediction. For example, robust projections of how a species will respond to climate change must consider how a species' response to a persistent increase in mean temperatures over many years will differ from the much larger---but short-term---fluctuations that many populations and species experience on a daily or weekly basis {\bf Figure 2}, and whether responses across such timescales are linked. Relatedly, given that most species ages are 1-10 million years \citep{lawtonbook} the best projections would also consider how a species has responded to previous major climatic shifts, which are often in equal magnitude and rate with climate change {\bf Figure 2}. Such examples highlight the two major dangers in temporal scaling: observations at too long a scale versus observations at too short a scale.\\

Observations collected at too coarse a temporal scale for accurate predictions are a common issue in ecology, as most ecological processes fundamentally operate on relatively short intervals (hours to days) that eventually scale up to longer time horizons (months to years). Yet, observations at longer timescales rarely scale down to shorter timescales: for example, the temperature sensitivity of ecosystem respiration derived from annual datasets does not reflect the short-term temperature sensitivity calculated by extrapolating from night to daytime data [Reichstein 2005 KM]. For projections, basal, physiological timescales are often more important for understanding ecological responses to the environment. Photosynthesis, for example, responds to daily variations in temperature and light availability (ref MANGROVE, XXXXX), which then integrates to gross and net primary productivity that will be additionally limited by weekly to monthly climate and weather variability (e.g., heat waves, frost events; ref XXXX). Phenology responds to climate at the daily timescale (daily heat accumulations, snowmelt, first rains of the wet season, photoperiod, etc.), but is critical for defining the beginning and end of the growing season (ref XXXX), setting fundamental limits on biological activity any given year. Yet, despite the recognized importance of these basal scales, many ecological models use monthly and seasonal climate predictors (ref XXXX), leading to a fundamental disconnect between the timescale of ecological theory and the timescale used in practice.\\

This issue is perhaps most clearly illustrated by classic climate envelope models that are often used to predict species distributions under climate change (ref XXXX). Such models typically use seasonal and annual average temperatures as the primary constraints on species ranges, but most evidence indicates that ecological processes---including species' ranges---are limited not by mean climate, but rather the recurrence intervals of extreme events (e.g., fire, droughts) or high order climatic moments (e.g., coldest winter day). For example, the distribution and population sizes of many insect pathogens are limited not by average summer or annual temperature controls on fecundity, but by mortality induced by minimum winter temperatures \citep{weed2013}. Further, increasing evidence from the paleorecord indicates that extant species have occupied quite different climate conditions in the past. Beetles that have survived several glacial/interglacial cycles in Europe had significantly different ranges and environmental conditions throughout the Quaternary \citep{Abellan2011}. This suggests that at the very least, a longer consideration of time would lead to extreme caution in the use of species distribution models under conditions very different than modern \citep[e.g.,][]{Williams:2007kx}. Robust projections of how a species will respond to climate change will also need to consider how a species' response to a multi-year increase in mean temperatures differs from much larger fluctuations that many populations appear buffered against on a daily or weekly basis {\bf Figure 2}. In phenology, for example, long-term warming trends are advancing spring events (e.g., earlier bud burst or flowering) in many temperate regions (ref XXXX), but may also be exposing these species to increased risk of damaging frosts \citep[e.g.,][]{Augspurger:2009gj}. Other species or ecosystems may show divergent responses to the same forcing occurring during different seasons, such as the tendency for many phenological events to be delayed by autumn warming but advanced by spring warming \citep[see][]{yu2010,Cook:2012pnas}. \\ 

A related downscaling issue for ecological forecasting comes from attempts in evolutionary biology to link information from phylogenetic trees---often spanning many millions of years---to species' modern-day responses to global change. For example, \citet{Wiens2005} suggested that shifts in the ranges of some species in
response to climate change reflect niche conservatism---the tendency of
species to retain aspects of their fundamental niche over time. This example
tacitly assumes information from two very different temporal 
scales---millions of years versus decades to centuries---can be
simply and directly linked. Making a similar assumption, \citet{lavergne2013} linked
rates of evolution to current sensitivity to recent environmental
change, with species within lineages characterized by more rapid rates of climate
niche evolution experiencing less severe demographic declines in recent times. However, efforts to
estimate evolutionary rates have been continually hampered by timescale issues
\citep{schoener2011,Uyeda2011}. Over short timescales observations
suggest rapid, but bounded evolution, consistent with population
divergence over a fluctuating adaptive landscape within an adaptive zone
\citep{Uyeda2011}. While, over the longer time-scales, sufficient for
speciation, variance increases slowly, but more or less
linearly---consistent with rare niche shifts that reconfigure the
adaptive landscape. Reconciling this apparent disjunct seems a primary
goal before phylogenetic information could be robustly incorporated into
ecological forecasting.\\

Rapid rates of evolution that reshape niches or lead to new species highlight an an important feature of temporal scaling---and one where temporal ecology diverges from spatial ecology---the presence in time series data of noncyclical and/or abrupt temporal patterns. Ecologists often refer to these as `events,' although that term itself has taken multiple meanings in the literature (see Box: Defining Events). `Events,' however, are a key feature of temporal ecology and clarification of the types of events can help identify the ecological processes involved. Within ecology, the term `events' typically refers to single, discrete occurrences, such as seed dispersal \citep{Higgins2003}, establishment of a new species \citep{Blackburn2011}, or an extreme forcing event such as a frost or drought \citep{Jentsch:2009ff} or a much larger shift, such as the Younger Dryas \citep{Jackson:2009el}. Here, we view events more holistically, and recognize that forcing events may be discrete (e.g., a frost or drought) or continuous (e.g., long-term warming trends), and the ultimate ecological responses may be transient (temporary) or continuous (persistent). Importantly, any ultimate ecological response to a forcing will depend critically on the system dynamics, including feedbacks (positive or negative), and any resistance or resilience inherent to the ecological system of interest. \\

The detection of an event depends on three sampling features: the temporal frequency of sampling, the duration of the sample period, and the magnitude of the event or departure from previous samples. Events may not be detected or perceived as events if any of these three features are insufficient (Figure 3). For example, sampling of plant species that flower at the beginning of a temperate growing season lead to the conclusion that the year 2012 was 3 standard deviations beyond the mean, and therefore 2012 could be considered an event (better label so this agrees with Box 1?). However, additional sampling of plant species that flowered later in the growing season demonstrates a less extreme response. Thus, the characterization of the year 2012 as an event that affected plant phenology would be dependent on sampling and the time period of interest. Similarly, a relatively cold interval in the middle of a warming trend could be either a late spring frost (one day), or the Younger Dryas climate event \citep{Jackson:2009el} depending on the duration of the event relative to the duration of the sampling period. \\

In addition to an improved appreciation and clarification of the role of events, several other major approaches in ecology can help address temporal scaling issues; in particular, we stress the very basic approach of leveraging complementary timescales offered by common ecological methods. Major approaches have often been abstracted into experiments, observations, long-term observations and modeling \citep{Carpenter:1992hk}. Experiments are often conducted on the shortest timescales---from days to weeks in the lab, to weeks and years in the field---and may only capture transient dynamics. They allow, however, generally the most powerful tests of mechanisms and, relatedly, major insights \citep{Chapin:1995af,Wolkovich:2012n,Laube2014}. Such powerful tests are buoyed by comparisons with observational data, even short-term (e.g., a single growing season or several years). Showing congruence with long-term data is a further boost, and using modeling to extend the scale of the findings---possibly incorporating important longer-term dynamics such as climate cycles---can greatly improve the utility of an original experiment as well as often providing additional insights and new predictions to test. Related to this, a greater incorporation of a longer temporal dimensions to short-term studies of long-term processes may come from incorporating perspectives of disciplines traditionally focused on long timescales. Today, ecology has a significant step-up on this from greater integration with other disciplines. Just as temporal scales inherently define and link subfields in ecology (Figure 2) they also bridge ecology to climate science, paleobiology, and evolution. Such links provide observational data from far longer timescales than usually available to ecologists and offer great opportunities to improve ecological forecasting by thoughtfully testing predictions across multiple scales of data. \\

For studies with only one major temporal scale of data, however, we stress that the key feature of appropriate temporal scaling is to match the timescale of observation with the timescale of the process of interest. Just as landscape ecology required identification of the relevant spatial scale for sampling, temporal ecology requires identification of the relevant temporal dimensions needed to address particular ecological questions. This means beginning with a list of all the relevant known timescales---including the generation times of the study organisms, frequency of disturbance, as well as the period of climate oscillations---then placing the study in the relevant part of these cycles. \\

Temporal scaling issues in ecology are clearly not new, nor are the above-described approaches to address them. A direct assessment of temporal scaling issues has, however, been unevenly applied across time and across subfields such that the inherent complexities time introduces to ecology have not always been transparent. A more even appreciation of temporal scaling in ecological systems will require continually cross-checking approaches of varying timescales, modeling studies to extend beyond currently available data, and more integration of disciplines that have sometimes worked separately because of their underlying disparate timescales (Figure 2). Such a shift is especially needed given the current rates of nonstationary processes, such as climate change, operating in many systems. \\

\noindent \emph{Nonstationarity in ecological systems}\\
\emph{Stationarity}, which refers to any stochastic process with a fixed, underlying probability distribution, is a major assumption of most statistical methods and many major concepts and theories in ecology. This assumption, however, is violated when the underlying probability distribution shifts across space or time: such \emph{nonstationarity} occurs often in ecology \citep{julio2012}, and can significantly limit the development and application of predictive models. In spatial ecology, there has been recent, increasing recognition of nonstationarity across space, leading to new hypotheses and methods. For example, geographically-weighted regression (GWR) relaxes the assumption that process parameters (e.g., variance and mean) are independent of location and direction \citep{Brunsdon:1998nx}, allowing researchers to test questions of whether differing processes---which shift across space---could shape fundamental ecological patterns, such as species richness \citep{davies2011}.\\ %  In temporal ecology, implicit assumptions of nonstationarity pervade certain arenas and underlie some of the discipline's oldest concepts, and debates. Yet ecological forecasting will require ecologists to more fully embrace how ubiquitous and important nonstationarity is to both abiotic drivers and biotic responses. % The concept of temporal nonstationarity is particularly critical for forecasting ...\\

Nonstationarity across time is also apparent in many aspects of ecological systems (e.g., disturbance, community shifts, extinctions and extirpations), yet ecology has an uneven history of embracing temporal nonstationarity. This is perhaps best illustrated by changing views on the concept of succession (changes in the structure and function of ecosystems over time), its predictability, and its relationship to the abiotic environment (Figure 4). Successional theories underlie one of the classic debates in ecology, pitting Clementsian versus Gleasonian versions of nature against one another \citep{clementsbook,gleason1926}. In the Clementsian version, succession is predictable (Figure 4a). A clear example of this view comes from studies of primary succession on newly-deglaciated surfaces in Glacier Bay, Alaska that describe a temporally-predictable sequence of four vegetation types: pioneer, Dryas, alder, and old-growth spruce and hemlock forests [W. S. Cooper]. Over this sequence, ecosystem properties changed over time \citep{Chapin1994}, with the rate (but not the endpoint) of succession controlled by biotic interactions and a minimal or non-existent role for the abiotic environment (e.g., climate). A classic alternative view of succession was offered by Gleason \citep{gleason1926}, who recognized the importance of the abiotic environment and expected far less predictable successional trajectories. This view includes recognizing that events such as disturbances could reset successional clocks (Figure 4b) and produce diverse ecological patterns across the landscape \citep{Levin:1992rg} (REF KENDRA). More recently ecology has recognized that communities may reach fundamentally different successional endpoints as a result of both biotic and abiotic interactions (Figure 4c), and that such endpoints may not be inherently predictable. Dendrochronological reconstructions of the vegetation history at different sites of the Glacier Bay sequence, for example, revealed at least three different primary successional sequences \citep{Fastie1995}. Such a view of succession is, by definition, not deterministic, but will depend on a broad suite of biotic (e.g., competition, dispersal and establishment rates) and abiotic (e.g., disturbances, climate change) processes that vary in their predictability and the degree to which we understand their effects on communities and ecosystems.\\
% Related to this, concepts of how systems respond to events, including definitions of resilience \citep{Grimm:1997} and multiple stable states \citep{yang2010} are fundamentally about how nonstationary systems are over the time period of interest. 

Modern ecology---with some exceptions---is dominated by the Gleasonian perspective, infused with a broad recognition of system states that are defined by nonlinear and irreversible trajectories \footnote{Kendra, add good ref?}. To develop skillful predictions, however, will require a better framework to integrate nonstationarity into ecological theory and concepts. This is particularly critical now, as systems are increasingly dominated by shifting anthropogenic factors such as habitat fragmentation, widespread dispersal of invasive and exotic species, and, perhaps most importantly, anthropogenic climate change---which represents a classic example of temporal nonstationarity (Figure 5a). For many systems, climate change has resulted in an obvious trend of increasing mean temperature, with variance about this mean often shifted equally \citep{Huntingford2013,Rhines2013}, though the concept of nonstationarity can result in shifts beyond the mean---including shifts in cycles or noise. \\

A framework for incorporating nonstationarity into ecological predictions, will require consideration of both nonstationarity in the forcings (e.g., climate change, Figure 5a), and also in the ecological responses (Figure 5b1,b2). Implications of such nonstationarity begin at the physiological level, where trends in climate may push species outside of their normal response range. For example, many species will advance their phenology with warming in a linear fashion until a certain threshold, after which responses are dominantly controlled by photoperiod or snow cover \citep{Iler2013} and may limit species' responses to further warming. Scaling up to the population and community levels, many current ecological models could be adapted to make predictions with climate change, if they were reapproached to relax assumptions of stationarity. Projections of shifts in communities with nonstationarity should benefit from increasing recognition of how temporal patterns structure ecological communities. For example, research on historical contingencies and, relatedly, priority effects may help forecast communities in nonstationary systems. Studies of community and ecosystem stability \citep{Boettiger2013}, paleoecological systems and modern disturbance ecology have provided foundational work on the role of contingency in driving ecological systems and highlighted that historical contingency is often more common than predictable, deterministic sequences over time. Moving beyond this recognition, the challenge to ecology now is to build theory that incorporates contingency in such a way that studies can provide more robust tests of how contingencies operate \citep{vannette2014}. For example, the role of multiple or compound disturbances in altering trajectories and whether environmental nonstationarity may make regime shifts more common by effectively removing the underlying environmental track a system was previously on (Figure 5c). The importance of historical contingency in ecology should not in any way suggest that systems are less predictable, it only highlights what ecologists already know so well---that ecological systems are complicated and accurate prediction requires careful knowledge of many driving factors.\\

Given the importance of nonstationarity, and the increasing evidence that ecological trajectories are not deterministic, ecological forecasting may benefit increasingly from probabilistic, rather than deterministic, modeling approaches, such as those used in the field of climate modeling \citep{Tebaldi2007}. Using a probabilistic approach allows for better elucidation of the internal, unpredictable variability in the system and uncertainties in various modeling approaches and parameter estimates. For example, many climate model projections use an ensemble approach, where individual ensemble members (i.e., climate projections) start from different initial conditions but use the same set of forcings and boundary conditions (e.g., land cover, greenhouse gas concentrations, etc). In this way a spread of projections is generated, with probabilities of future climate states emerging naturally from the ensemble as a function of the forced response (ensemble average) and internal variability (ensemble spread). Probabilistic sampling and modeling may also allow for detection and attribution of controversial topics in ecology for which data are limited, such as $CO_{2}$ fertilization and invasive species, as well as understanding the importance of very rare, but random events, like Black Swans (Box 1).\\
% Lots of ecology is nonstationary, In contrast, many underlying assumptions of ecological theory and basic approaches are not.\\
 
\noindent {\bf Space-time}\\

A more routine incorporation of temporal scaling and nonstationarity into ecological studies will allow ecology to tackle its next challenge: to unite predictive frameworks from spatial and temporal ecology. Recent advances in temporal ecology have come from the rise of spatial ecology and its emphasis in recognizing and understanding hidden dimensions in ecological models and theory. Decades of efforts in understanding the consequences of when space is made explicit have resulted in a return to the importance of temporal dimensions of many spatial models. For example, island biogeography theory predicts species richness based on several basic spatial metrics---but temporal dimensions of the controlling processes---immigration, extinction and speciation---also clearly govern predictions of species richness \citep{Wiens2011}. Similarly, disease models have advanced through incorporating both spatial and temporal models of traveling waves \citep{Grenfell:2001ox} as disease prevalence varies both with population density and temporal fluctuations in that density \citep{Grenfell:2001ox} and climate \footnote{JD, please suggest ref}. Paleoecological data spanning 10,000 years is being gathered into databases for spatial analysis, so that past vegetation assemblages can be visualized in both space and time over much of eastern North America [Brewer et al. 2010 KM -- need ref]. Perhaps the current best example of space-time integration comes from outside of ecology, and instead from climatology, where empirical orthogonal function (EOF) analysis helps to identify at once temporal and spatial patterns in climate data. While currently not employed in ecological analyses some data, such as long-term population monitoring, are now approaching the duration and frequency that would allow similar approaches in ecology. \\
% cut example: At the ecosystem-scale, spatial and temporal patterns of ecosystem carbon flux are being studied using a network of 253 eddy covariance research sites in the FLUXNET global database. With nearly 1000 site-years of data, low frequency and high frequency signals in the climate signal are being identified so that the net effect of climate change on ecosystem carbon exchange can be isolated [Stoy et al. 2009 KM]. 
% For example, island biogeography can help make predictions of species richness, but assumes stationarity in the mainland population even though few mainland populations in recent times are stationary. 

Such advances represent, however, only a small foray into the applications of fully embracing the interconnectedness of spatial and temporal dynamics in ecology. Coexistence theory---long stymied by models that required \(n\) different axes to produce \(n\) coexisting species alongside empirical examples of many co-occurring species that appeared quite similar when examined from one snapshot in the lab or field---advanced when the role of variability in species responses to the temporal dimension was re-examined \citep{Chesson:1997dz}. Under the storage effect model \citep{Chesson:1997dz} highly similar species can co-exist via small differences in how they respond to temporal variability in the environment. Since its introduction the storage effect model has been ported to spatial dimensions as well---where species coexist via reduced competition from spatial variability. Tests for such models have found support separately for temporal \citep{Angert:2009} and spatial \citep{Sears:2007md} storage effects in studied communities, but we expect most communities function based on a constantly shifting mix of the two mechanisms (in addition to fluctuation-independent mechanisms). For example, studies of community change during the Dust Bowl show how dramatically dominant species may decline to apparent local extinction while rare species rise to abundance \citep{Weaver1936}. Recent work, however, highlights that the persistence of rare species throughout long time periods may come from landscape dynamics where microclimates maintain great species diversity \cite{Craine2012}. In such cases temporal storage effects are built on buffered population growth maintained by spatial dynamics. Further, by modeling the environment explicitly, such models could make predictions of how fundamental coexistence mechanisms may shift with climate change and help answer the critical question of what happens to communities built on coexistence mechanisms via a temporally and/or spatially variable environment when that environment switches from stationary to nonstationary. Additionally, robust projections of climate change impacts on populations and species will require a nonstationary adjustment to the most classic spatial metaphor for a temporal process: adaptive landscapes. Ignoring species interactions and considering only shifts to the abiotic environment, climate change has resulted in rapid and effectively continuous shifts to most populations adaptive peaks and valleys. Climate change has thus highlighted how rapid evolution may be and has brought it firmly into an ecological timescale, but theory to how such nonstationarity may affect evolutionary outcomes has not kept pace \citep{schoener2011}. \\ % THIS HIGHLIGHTS JUST A TASTE OF WHERE WE COULD GO. %Given the tremendous diversity of temporal variation in most environments it should not be surprising that it would be a major axis on which to mute competitive forces and thus promote coexistence.

\noindent {\bf Conclusions}\\

Over thirty years ago, habitat fragmentation led to the emergence of spatial ecology as a unifying framework for analyzing and interpreting spatial patterns across the disparate subfields and disciplines of ecology. Today, a similar structure is beginning to develop around the concept of time, introduced in this manuscript as temporal ecology, and motivated primarily by the need to better understand the direction, rates, and predictability of ecological responses to climate change. But even as a rapidly growing body of work is being developed to inform how species and communities will respond to increasing rates of anthropogenic climate change, general theories and paradigms to shape and guide these studies have yet to fully emerge. We believe a new framework is needed that parallels efforts in spatial ecology improving and developing methods, building on basic theories and concepts but structured around time. This temporal focus will require integration across timescales, disciplines, and methodologies, borrowing from the fields of evolution, ecology, and climatology. Recent advances across subfields in ecology incorporating environmental variability into coexistence models \citep{Chesson:1997dz}, bridging ecological and evolutionary timescales \citep{schoener2011}, revisiting the role of climatic events in setting range limits \citep{Tran2007} and in modernizing paleoecology \citep{Brewer2012} all suggest modern ecology is up to the challenge.\\ % Whoo! Ben is so good at being positive.

Such a temporal ecology framework would be especially valuable for conservation science in the Anthropocene---where ecological dynamics operate in an increasingly nonstationary world, dominated by rising rates of anthropogenic change. Traditionally, conservation biology focused on space---that is, identifying the best locations to conserve species or habitats under threat from habitat change and loss \citep[e.g.,][]{Doak:1989oc,Cincotta:2000gk}, motivating the establishment of reserves, refugia, and corridors \citep[e.g.,][]{Doak:1989oc}. But while the ecological consequences of habitat loss may be more obvious or immediate, climate change highlights the necessity of also considering changes over time \citep[e.g.,][]{Laurance2012}. For example, tree line shifts associated with climate change may shift habitat suitability for some species, suggesting that conservation areas should be set not only for the current suitable areas, but also for how these areas might change in the future. A joint consideration of space and time will thus help resolve some of the current debates on tradeoffs between prioritizing species conservation for habitat loss (space) and climate change (time), and a dual consideration of both space and time will allow the identification of where and when the best opportunities exist for mitigation and conservation.\\

Furthermore, a broader temporal ecology perspective may also help inform the probability and potential impact of extreme events, such as black swans, and the resistance and resilience of ecosystems to these events. For example, a species or ecosystem may easily adapt to long term changes in the average climate (e.g., long term warming), but recent events in many landscapes \citep{Anderegg2013} highlight that the frequency and impact of extreme events (drought, insect irruptions) may fundamentally shift baselines. This suggest conservation strategies must consider the impact of such events and how they may impact the resistance and recovery of ecosystems to further events in the future. Insights into these issues can be gained from historical and paleoecological data but---again---any future projections or plans will require recognizing the nonstationary nature of these processes. Conservation would thus benefit from both spatial and temporal considerations that may require, for example, establishment of a reserve with diverse topographic landscape or including migration corridors, accounting for diverse possibilities (however unlikely) in order to allow species to better adapt to extreme events in the future. [Cut/fix last sentence?]\\

Clearly, ecology has advanced significantly in recent decades as ecological data spanning years, decades, and centuries have become increasingly available for paleo-, conservation, community, and ecosystem ecology applications. The challenge remains, however, to develop a holistic structure that will allow for cross-disciplinary sharing of methods and ideas and leverage the strengths of these disparate fields. Encouragingly enough, such work is rapidly being developed in areas like phenology \citep{wgreview}, paleoecology \citep{Brewer2012}, and even in conservation \citep{mooers2008}, pointing to the potential for rapid advances and developments in the field of temporal ecology in the coming years.\\

\noindent {\bf Acknowledgments}\\
Conversations with D. Schluter, M. Donohue and S. Brewer helped this manuscript and its bumbling underlying concepts. NSERC BRITE funding. % We thank both Monsters and Men. 


\newpage
\noindent {\bf Defining events}\\
\noindent With the increasing availability of long-term environmental and ecological data from the historical period (DEFINE) and paleorecord, ecology is better poised now than ever before to improve understanding of temporal dynamics in ecological systems. One area of temporal dynamics that is of particular interest is how quickly and persistently ecosystems respond to forcings that may be either short-lived, \emph{transient} events (e.g. fires, droughts, insect defoliation, etc) or persistent, \emph{continuous} changes in the background state (e.g., climate change, introduction of invasive species, habitat fragmentation, etc, see Figure). Importantly, the permanence and velocity of ecological responses depends not only on the nature of the forcing (e.g., its severity and duration), but also on the internal dynamics and feedbacks within the ecosystems or communities themselves.\\

\begin{figure}[h!]
\centering
\noindent \includegraphics[width=0.4\textwidth]{/Users/Lizzie/Documents/git/manuscripts/temporalecology/figures/events/events.png}
% \caption{}
\end{figure}

Forcings in ecological systems may be discrete, transitory events such as fire or insect defoliation events, but the ecological consequences may be transient and short-lived [1], or persist long after the forcing has disappeared or reversed [2]. Conversely, the forcings my be continuous and persistent, such as global climate change or the establishment of an invasive species. Similarly, however, these continuous forces may give rise to transient ecological responses [3] or continuing responses [4]. Whether the ecological response is transient or continuous depends not only on the nature of the forcing, but also on the inherent capacity for resistance, resilience, and feedbacks within the ecosystem or community of interest.\\

For example, vegetation may quickly return to its previous state following transient disturbances, such as a fast growing grassland recovering after a fire or drought \citep[e.g.,][]{Weaver1936,albertson1944}, or a plant down-regulating initial photosynthetic enhancement in response to elevated $CO_{2}$ concentrations \citep{eatonrye2012}. Both responses can be considered transient, regardless of the nature of the forcing, and may indicate either some inherent resilience in ecosystem structure and function (in the grassland example), or fundamental shifts in the in the importance of the resource limitation and environmental stressor space. Ecosystems may also respond in persistent ways to either transient or continuous forcings. A relatively recent example is the switch from a ponderosa pine forest to a piñon–juniper woodland in Southwest North America following a major drought in the 1950s \citep{Allen1998}. This new woodland persists to this day, despite a subsequent return to more normal moisture conditions. And during the Mid-Holocene, the Sahara permanently shifted from a woodland savanna to a hyper-arid desert in response to changes in Northern Hemisphere summer insolation, with the ecosystem collapse happened much more quickly than the forcing change \citep{Foley2003}. Clearly, the nature of forcing events (fast or slow, discrete or continuous) does not necessarily map clearly onto ecological responses, presenting a challenge for better prediction of the speed and persistence of ecosystem responses.\\

Additional difficulties may be presented by a special class of events known as `Black Swans'. A Black Swan event is defined by two components: (1) that it has dramatic effects on the system, but is extremely rare, such that (2) it is effectively impossible to predict using current methods. These two components lead to the third aspect of Black Swan theory: owing to a combination of their unpredictability and their large impact on the system there is a strong tendency to believe such events can be predicted---when, instead, by their extreme rarity and unpredictability this is generally impossible. There is already evidence for ecologically important `Black Swan' events. One example is an 18th century drought in Eastern North America that has shaped successional trajectories to this day \citep{Pederson2014}. Another, more well known example, is the Salton Sea, an inland body of water in southern California that formed during a large flood event in the early 20th century, becoming a critical habitat for wildlife and migratory birds \citep{Cohn2000}. Identifying these events and their importance for ecological processes in historical and paleoecological data, however, remains challenging.\\ 

% (Moved up currently.) To achieve the goal of better understanding the importance and time scale of forcings and ecological responses, ecological modeling may benefit increasingly from probabilistic, rather than deterministic, modeling approaches, such as those used in the field of climate modeling (ref XXXX). Using a probabilistic approach allows for better elucidation of the internal, unpredictable variability in the system and the full exploration of the possible model parameter space. For example, many climate model projections use an ensemble approach, where individual ensemble members (i.e., climate projections) start from different initial conditions but all use the same set of forcings and boundary conditions (e.g., land cover, greenhouse gas concentrations, etc). In this way a spread of projections is generated, with probabilities of future climate states emerging naturally from the ensemble as a function of the forced response (ensemble average) and internal variability (ensemble spread). Probabilistic sampling and modeling may also allow for detection and attribution of controversial topics in ecology for which data are limited, such as CO2 fertilization (ref XXX) and invasive species (ref XXXX), as well as understanding the importance of very rare, but random events, like Black Swans.


\newpage
\begin{footnotesize}
{\def\section*#1{}
\bibliography{/Users/Lizzie/Documents/EndnoteRelated/Bibtex/LizzieMainMinimal}
}
\end{footnotesize}

\newpage

\begin{figure}[h!]
\centering
\noindent \includegraphics[width=0.5\textwidth]{/Users/Lizzie/Documents/git/manuscripts/temporalecology/figures/prediction/prediction.png}
\caption{Increasing availability of long-term records in ecology means many labs now possess repeated measures observational data of 5-10 years or more. Such data provide ecology with a great opportunity to improve understanding and prediction, but also challenges researchers with how best to interpret trends in such data. Depending on the system and period of observation, what looks like a linear increase (a) could be part of a regular long-term cycles (b), indicative of a major shift in the system into a nonstationary period (c) or possibly part of both (d), especially if forcing on the system has changed---as seen in many systems with climate change. Temporal scaling and nonstationarity are, thus, inherently linked as anything can look stationary or nonstationary depending on the scale.}
\end{figure}

\newpage
\begin{figure}[h!]
\centering
\noindent \includegraphics[width=1\textwidth]{/Users/Lizzie/Documents/git/manuscripts/temporalecology/figures/ecoevoclimate/climateEcoEvo_alltogethernowPlus.png}
\caption{Robust forecasting in temporal ecology requires recognizing the multiplicative dimensions of time inherent in most ecological processes (top arrows). For example, predictions of species' responses to climate change must consider: (1) that many species experience far larger shifts in temperature on the timescale of hours to days (left) and (2) that over their evolutionary history many species have experienced climate swings similar in magnitude and rate to current and projected anthropogenic climate change (middle), in addition to the pressures of glaciation cycles (right). Differing methods in ecology (bottom) are optimized to differing timescales but ecologists are generally most adept at working in timescales of days to years.}
\end{figure}

\newpage
\begin{figure}[h!]
\centering
\noindent \includegraphics[width=0.8\textwidth]{/Users/Lizzie/Documents/git/manuscripts/temporalecology/figures/extremes/extremes.png}
\caption{Both the temporal frequency (a) and duration (b) of sampling affect the utility of time series data to understand how climate impacts ecological dynamics. In plant phenology research (a), within-year sampling is often focused on the start of spring and a small number of species, with fewer observations of species that flower later in the season. Increasing the frequency of sampling such that more species are studied across the growing season can affect the interpretation of a climate event. For example, 2012 was an extremely warm spring for much of northeastern North America and early season species, such as \emph{Prunus serotina}, flowered much earlier than ever previously observed ($z$-scored data show 2012 was three standard deviations beyond the mean for this species, inset shows raw time series data), while many other species, such as \emph{Betula lenta} that flower several weeks later, did not show as extreme a response. Duration of sampling can also easily affect how extreme an event---such as a drought (b)---appears. Considering only the 20th century (b, see light blue bars in histogram, and light blue shaded area of inset), a 4-year drought in central North America appears uniquely long. Extending the record, however, to the last 1,000 years (gray bars in histogram) shows far longer droughts throughout the 12th to 15th centuries. See Online Supporting Information for details on data.}
\end{figure}

\newpage
\begin{figure}[h!]
\centering
\noindent \includegraphics[width=0.4\textwidth]{/Users/Lizzie/Documents/git/manuscripts/temporalecology/figures/assembly/assembly.png}
\caption{The evolution of how ecologists view time can be seen partly in the maturation of successional theory. Early work (a) tended to focus on one possible trajectory and outcome. A consistent, predictable turnover of species was the main driver of ecosystem development; mean climatic factors shaped the species pool on the largescale, but climate was otherwise generally unimportant. As work progressed (b), ecologists recognized that multiple trajectories were possible---often triggered by disturbances that fundamentally reset the temporal position of an ecosystem along its development curve. Disturbances include a variety of factors related to climatic extremes (e.g., drought, fire) and their inclusion in theory highlighted a growing recognition that climate variability could impact succession. More recently (c), ecologists have layered onto this an appreciation of factors that may yield diverse trajectories and endpoints. Research on priority effects and other small discrepancies in initial conditions between otherwise similar sites have shown that such differences can result in different trajectories and endpoints. Additionally, research on tipping points and alternative stable states have highlighted that some events may transition ecosystems to fundamentally different states; nonstationarity in climate, or other ecosystem drivers, may contribute to such tipping points as it could be difficult for systems to return to trajectories if the underlying climate has shifted significantly while the system recovers from the disturbance.}
\end{figure}

\newpage
\begin{figure}[h!]
\centering
\noindent \includegraphics[width=1\textwidth]{/Users/Lizzie/Documents/git/manuscripts/temporalecology/figures/nonstationarity/nonstationarity.png}
\caption{Forecasting ecological responses to climate change requires layering projections of complex physiological responses onto nonstationary drivers, such as increasing temperatures (in a, we show annual northern hemisphere temperature for ocean and land shown with a 10-year locally weighted scatterplot smoothing in red). Much research has focused on projecting phenological responses to this increasing temperature (b), which requires understanding whether responses are fundamentally linear (b.1), where higher spring temperatures yield earlier leafing or flowering or non-linear (b.2) where responses to temperature are limited at higher values by additional factors; here we consider a highly simplified photoperiod limit as an example, though drought, nutrients or other factors could also be critical. Such responses may look identical under a stationary climate regime (lighter blue shading), but would become apparent once the threshold is crossed under nonstationary climate (darker blue shading).}
\end{figure}



\end{document}

\noindent \emph{Text parking lot:}\\
It has provided a major new impetus to revisit fundamental questions whose answers have applications far beyond climate change: for example, understanding how environmental nonstationarity affects coexistence via the storage effect has implications for questions of how and when the storage effect may evolve and how species survive glaciation and other major climatic events. 


Nonstationarity in {\bf drivers}
\begin{enumerate}
\item Examples? $CO_{2}$ -- which is trend in mean, some example of shift in variance or such ...
\item Anything else here?
\end{enumerate}

Stationarity and nonstationarity in {\bf responses}, see {\bf Figure 4}
\begin{enumerate}
\item linear vs. nonlinear response functions (example with phenology, Primack vs. vernalization etc.) $\rightarrow$ tie to physiological level (err, now we're in scaling world)
\item species have different responses (nonstationarity across species? ack.)
\item species-specific responses should link up to how nonstationary drivers affect drivers of coexistence .... 
\end{enumerate}



%% 
%% Not sure if this fits somewhere (the evolving view of Chessonian coexistence? Or really may fit below)
This shift in perspectives on succession is one example of ecology's maturing view of the complexity of community assembly (Figure 2 `previous'). Early views of succession effectively assumed a simple environmental filter where climate was implicitly assumed to be stationary background noise, never providing any means to shift or alter trajectories; layered on this was the Hutchinsonian niche model where species coexisted via separate hyperdimensional moulds of resource use (which were often implicitly assumed to be effectively constant in space and time). A more modern view of succession includes an environmental filter that considers climate means as well as regular oscillations and events that may reset succession (Figure 2 `recent'). While researchers still sometimes use climatic niche distribution models to estimate environmental filters these is growing return to basic physiological limits that truly define the environmental filter for most species. Such physiological limits are most often defined by climate events and it is here that climate projections are moving. \footnote{Lizze: Add in Ayres beetle example.} Layered on this is increased recognition of the important variability in climate to coexistence: the Hutchinsonian niche is now one considered mode of coexistence, with increasing research in Chessonian coexistence mechanisms such as the storage effect highlighting the importance of temporal variability in shaping the biotic filter of community assembly. Chessonian coexistence has, in some ways, blurred the lines between the environmental and biotic filters of community assembly. \\

Understanding how both temporal and spatial dynamics may control communities is an important step toward climate change projections, but will critically require an additional layer of understanding how nonstationarity will shift these dynamics. The Dust Bowl was a major event for tallgrass plant communities---and typifies an important difference between many previously-studied climate events and the non-stationarity of current anthropogenic climate change. Following the Dust Bowl as climate returned to a state similar to that before the Dust Bowl, many plant communities also eventually returned to species assemblages similar to those before the Dust Bowl \citep{Weaver1936}. This is a rough prediction of the storage effect model, which assumes underlying stationarity in the environment and species responses to it. Climate change, however, is specifically a nonstationary event and predictions of what happens to communities under nonstationary environmental dynamics are not well-developed (see next section on `Contingency'). 
%%

% OLD 

While succession highlights that most ecological systems are inherently nonstationarity across long (century to millenia) timescales, many anthropogenic forces affecting ecological systems---for example, fire regimes that shift due to human ignition sources, damn-building phases in step with regional and country-level development---are nonstationary over much shorter timescales. Recent climate change, at least partially associated with greenhouse gases \citep{ipcc2013}, represents an excellent example of nonstationarity (Figure 6). For many systems, climate change has resulted in an obvious trend of increasing mean temperature, with variance about this mean often shifted equally \citep{Huntingford2013,Rhines2013}, though the concept of nonstationarity can result in shifts beyond the mean---including shifts in cycles or noise. \\

% Chronosequences
Several major approaches in ecology address temporal scaling issues. The first and perhaps most-widely applied method substitutes space for time (chronosequence, CITE Jenny 1941). Research in long-term ecosystem development relies on this approach [Jenny 1941], examining how the state factors of climate, organisms, relief, and parent material vary across space to understand long-term shifts in ecosystem properties, such as net primary productivity \citep{Wardle:2004wb}. This type of approach makes several major assumptions---including stationarity in the trajectory of ecosystems across space (see Figure X or some section below)---and requires extensive effort to locate relevant sites and pinpoint their temporal dimension, but provides tremendous power to ask questions on timescales far longer than the full lifespan of the field of ecology and, perhaps most powerfully, to project forward. 
% end % 

% JD's bunny story:
You jest, but this is perfect! Myxomatosis. We have a mix of slow and fast processes, rabbit generation times, and population cycles versus viral replication rates (http://www.jstor.org/stable/1943014).

Virus originally introduced into Australia  around 1950, initially had 99\% mortality rate, and at some sites rabbit pop dropped by 90\%. However, within a year mortality rate had dropped to 90\% and continued to decline and was <30\% within 8 yrs.

The increased resistance of the rabbit was due to the disease-induced mortality of genetically susceptible rabbits (i.e. selection on rabbit population), while the decline in the frequency of the more virulent strains of the virus depended on the interaction between the virulence of the virus and its transmissibility by the vector (mosquitos) - strains which were of intermediate virulence were transmitted more (too virulent and you kill the rabbit, not virulent enough and you produce too few viral particles). Emergent property of intermediate virulence as a function of viral evolution and selection on rabbit populations. In addition, results in periodic cycles in rabbit populations (from simulations) reflecting seasonal reproductive rate and host-pathogen dynamics (fig 6 in paper).

% JD's GWR stuff:

Spatial processes also demonstrate nonstationarity. Ignoring nonstationarity can misinform predictive models, which typically assume the parameters of a process (e.g. variance and mean) are independent of location and direction (Fortin and Dale, 2005). However, relaxing this assumption, for example, using geographical weighted regression (Brunsdon et al. 1996: http://onlinelibrary.wiley.com/doi/10.1111/j.1538-4632.1996.tb00936.x/pdf), which allows different relationships to exist at different points in space, can provide novel insights, helping discern process from pattern. GWR has proven particularly informative in the field of spatial macroecology as we have better mapped the distribution of animal and plant species richness across the surface of the Earth, and identified the major environmental axes with which it covaries. The latitudinal gradient in species richness is perhaps the most well-known biodiversity pattern, and the search for the underlying driver has been referred to as the holy grail of ecology (Huston M. A. 1994 Biological diversity: the coexistence of species on changing landscapes. Cambridge, UK: Cambridge University Press.). Competing explanations suggest the tropics as either a museum of diversity, where species richness has accumulated gradually over long times, or a cradle of diversity, in which high tropical species richness is explained by rapid diversification (Stebbins 1974: Stebbins G. L. 1974 Flowering Plants: Evolution Above the Species Level Harvard Univ. Press, Cambridge, MA). Fossil evidence suggests that the tropics might act as both a cradle and a museum (Jablonski et al. 2006: http://www.sciencemag.org/content/314/5796/102.short). By exploring non-stationarity in environmental predictors it is possible to reveal different richness predictors in the New World versus the Old World, suggesting explanations for the latitudinal gradient in species richness might differ between these two major biogeographical regions (see Davies et al. 2011; doi: 10.1098/rstb.2011.0018).
