\documentclass[11pt,a4paper]{letter}
\usepackage[top=1.00in, bottom=1.0in, left=1.1in, right=1.1in]{geometry}
\usepackage{graphicx}

%\signature{}

\begin{document}
\begin{letter}{}
\includegraphics[width=0.4\textwidth]{/Users/Lizzie/Documents/Professional/images/letterhead/arnold/AASmLogo2colr.jpg}

\opening{Dear Dr. Bascompte:}

\noindent Please consider our manuscript `Temporal ecology in the Anthropocene' as an Ideas and Perspectives piece in \emph{Ecology Letters}. We have previously corresponded with you regarding this and received approval to submit this manuscript for your consideration.\\
\vspace{-1ex}\\
The two greatest threats to ecological systems in the Anthropocene---habitat degradation and climate change---represent human modification of space and time. This manuscript argues that climate change---the effective manipulation of time by humans---has generated a current need to build a more predictive, integrative and interdisciplinary framework for temporal ecology. We contrast this with the manipulation of space by humans, through habitat loss and degradation, and the rise of spatial ecology. To lay the groundwork for this new framework we focus on similarities between time and space, namely scaling issues related to grain and extent, as well as critical differences between time and space, especially the directionality of time and the importance of events. Focusing on climate change, we place special emphasis on nonstationarity in the environment and its impact on ecological data, concepts and approaches. Though we focus on climate change as our main example, we also examine how other anthropogenic forces have also altered temporal dynamics. Further, though our focus is on applied issues, we stress that---like spatial ecology---temporal ecology is fundamentally aimed at addressing some of ecology's most basic questions.\\
\vspace{-1ex}\\
This manuscript benefits from the perspectives of a diverse authorship group including experts in climate science, climate change and community ecology, evolution and paleobiology. Although we are aware that the field of temporal ecology could encompass many more subdisciplines of ecology, we believe we have focused on important areas within our personal specialties and have integrated perspectives from many additional subdisciplines. The manuscript thus covers a breadth of topics, including adaptive landscapes, coexistence theory, succession, and ecosystem development. We have worked to have both breadth and depth---picking what we consider the most important and illustrative theories and examples. Additionally, we have focused on providing---where possible---quantitative data to make our arguments: our figures include a diversity of time-series data (Figures 1, 4, 7) and an extensive coalescence of data to contrast rates of climate change over different timescales (Figure 5).\\
\vspace{-1ex}\\
\emph{Novelty:} While ecology has long embraced the importance of temporal dynamics, our manu- script is the first to suggest explicitly how anthropogenic climate change has highlighted weaknesses in our framework for temporal ecology, and to lay the groundwork for an improved framework. We believe our comparisons to spatial ecology are unique and will aid development of this new framework. In particular, we believe our treatment of nonstationarity is not only novel but critical today as much environmental time-series data are nonstationary, yet ecology has not fully embraced the concept and its implications throughout its methods, theories and concepts. The authors on this paper have all published on temporal concepts---including long-term perspectives on climate, paleo-ecosystem dynamics and evolution as well as work on climate change and phenology---however, none have any similar work on the theme of temporal ecology.
\\
\vspace{-1ex}\\
We suggest the following as possible reviewers:
\begin{itemize}
\item Julio Betancourt, USGS National Research Program, jlbetanc@usgs.gov
\item Brian McGill, University of Maine, mail@brianmcgill.org
\item Kendi Davies, University of Colorado, Kendi.Davies@Colorado.EDU
\item John Dearing, University of Southampton, J.Dearing@soton.ac.uk
\end{itemize}
Drs. Doug Bolger, Mary O'Connor and Dolph Schluter have previously read, and helped improve, the manuscript. 
\\
\vspace{-1ex}\\
All authors substantially contributed to this work and approved of this version for submission. All data used in figures are publicly-available already and we provide appropriate citations. The manuscript has a 198-word abstract, a main text of 6,027 words, 79 references, 7 figures, and one box. There is only one recent paper by a co-author that we cite (Pederson \emph{et al., in press}). We attempted to secure a preprint to submit, however, the paper is embargoed and the first author would prefer to wait to share a copy until the embargo is lifted (16 April). This cited paper, however, is a data paper that we use as one example (of many) and is not related to or similar to the manuscript we are currently submitting. We would be happy to send along a preprint of Pederson \emph{et al., in press} next week if needed. 
\\
\vspace{-1ex}\\
This manuscript is not under consideration elsewhere. I hope that you will find it suitable for publication in \emph{Ecology Letters}, and look forward to hearing from you.
\\
\\\vspace{-1ex}\\
\noindent Sincerely,\\

 \includegraphics[width=0.3\textwidth]{/Users/Lizzie/Documents/Professional/Vitas/Signatures/SignatureLizzieSm.png} \\

\noindent Elizabeth M Wolkovich (on behalf of my co-authors)

\end{letter}
\end{document}



